\errorcontextlines 999999
\documentclass[userip,english,cpalette=Crimson]{lecture-fancy-paper}
\PassOptionsToPackage{threshold=1}{csquotes}
\usepackage[%
    sopra-listings={cpalette,numinpar},%
    sopra-tables={cpalette},%
    sopra-models,%
    lecture,%
    lecture-bibliography={biber,style=numeric},%
    lithie-boxes={germanenv,autostyle},%
    lithie-task-boxes={cpalette},%
    lecture-links={patchurl}%
]{lithie-util}

\input{color_definition.src}
\pagecolor{white!97!lightgray!99!paletteA}

\usepackage[style=mmddyyyy,yearmonthsep={/},dayyearsep={/},monthdaysep={/}]{datetime2}

\LoadBoxStyle{limerence}

\addbibresource{bibliography-collection/lithie.bib}

\makeatletter
\def\input@path{{source/}{figures/}}
\makeatother

\loadglsentries{glossary}
\usepackage{wrapfig,lipsum}
\usepackage[imgprefix={figures/asq_},command={Asq}]{lecture-personal}

\title{Limitations of science}%
\subtitle{The inexplicable, god, and its development in science}
\def\titleOfASQ{20\textsuperscript{th}~Century Thinking about Science}
\brief{\disablehyper\titleOfASQ~(\uc{asq})\\*\typesetAuthor, \lectureGetRegister{date}}
\addAuthor{Florian Sihler (florian.sihler@uni-ulm.de)}
\date\DTMtoday
\subject{ASQ, Elaboration}
\supervisor{Dr. Hans-Peter Eckle}
\authority{Ulm University, 2021}
\copyright{\href{https://creativecommons.org/licenses/by-nc-sa/2.0/}{CC BY-NC-SA 2.0} \faCreativeCommons\ \faCreativeCommonsBy\ \faCreativeCommonsNcEu\ \faCreativeCommonsSa}

\usepackage{blindtext}

\newcommand*\footurl[3][]{\footnote{#1\url{#2}\quad(\DTMdate{#3})}}
\def\UrlBreaks{\do\/\do-}
\newcommand*\shortfooturl[4][]{\footnote{#1\href{#2}{#3}\quad(\DTMdate{#4})}}
\newcommand*\shortfootarchiveurl[4][]{\footnote{#1\href{#2}{#3}\quad(archived, \DTMdate{#4})}}
% \;\smash{\raisebox{0.3\baselineskip}{\scalebox{0.35}{\faUniversity}}}

\usepackage{pbox}
\def\flocomment#1{\fcolorbox{paletteA}{shadeA}{\pbox{\linewidth}{\sffamily\textbf{\paletteA{Flo:}}~#1}}}

\def\formatshort#1#2{#1~\smash{\raisebox{0.25\baselineskip}{\resizebox{!}{0.25\baselineskip}{\pgfornament{14}}}}~#2}

\newcommand*\bce[1]{#1~BCE} % keep consistent with bibrefs
\SetBlockThreshold{0}

\usepackage{newunicodechar}
\newunicodechar{Ṭ}{\d{T}}
\newunicodechar{ṭ}{\d{t}}
\newunicodechar{ṣ}{\d{s}}
\newunicodechar{Ḥ}{\d{H}}
\newunicodechar{ī}{\=\i}
\newunicodechar{ā}{\=a}

\renewcommand*{\bibfont}{\normalfont\small}


\begin{document}

\maketitle
% \lecturedivider

\begin{abstract}
    Since the dawn of mankind, humans are trying to understand how the world works.
    Their explanations, especially at the beginning, were dominated by a divine view of the world, which was supplemented by more and more science-based knowledge over time.

    This elaboration gives a brief overview of the history of the most important changes in the scientific worldview and their limitations, focusing on western cultures.
    Furthermore, the meaning of those limitations will be discussed, using the relationship to a god of important scientific-contributors.

    It turns out that an (even imagined) limit in the perceivable worldview is not beneficial to scientific research. Also, the integration of religion into science even seems to be beneficial.
    % and in no way can science ever fully replace religion
\end{abstract}

This work was created as an elaboration for the \uc{asq}: \say{\titleOfASQ} by \lectureGetRegister{supervisor} in the winter-semester 2020/2021 at the University of Ulm. The \LaTeXe-Documentclass was custom-made for this document by the author.

\section{Introduction}

When talking about the origins of science, Aristotle is one of the most important figures, shaping the basic cycle of induction \& deduction and laying the groundwork for western science.\footurl{https://www.britannica.com/biography/Aristotle}{2021-01-16}
While Aristotle tried to fathom the world and its causal workings, he still believed in a god/in something divine.
His famous work \say{Metaphysics} establishes the science of the divine (theology) as one of the three important pillars, next to the ontology and the science of general principles~\cite{aristotle350}.
However, at such an early stage it may seem logical to use belief  (as belief in something higher, something divine) to fill in the blanks that science is unable to explain (yet).

These days we are much more enlightened and informed about how the world works and how we, as humans, work, questioning the need for belief.
Physics, biology, and chemistry, together with mathematics and especially with the information-technology revolution, opened up areas that were unthinkable at the time.
Some of those discoveries, such as Galileo's heliocentric worldview or Darwin's theory of evolution, can even be equated with blasphemy from the perspective of the (Christian) church, at least at the time they have been discovered.

Still, a lot of modern scientists, like Sir Isaac Newton~\cite[p. 315]{westfall1983} or Jérôme Lejeune,\footurl{https://lejeunefoundation.org/jerome-lejeune/}{2021-16-01} do believe in a god (or a higher power in general), others like Richard Feynman~\cite{feynman2001,brian2001} and Steven Hawking\shortfooturl{https://time.com/5199149/stephen-hawking-death-god-atheist/}{\formatshort{https://time.com}{5199149}}{2021-01-16} do not.
To quote the latter, the following quote expresses Hawking's opinion quite well:
\blockquote[Stephen Hawking\shortfooturl{https://abcnews.go.com/GMA/stephen-hawking-science-makes-god-unnecessary/story?id=11571150}{\formatshort{https://abcnews.go.com}{11571150}}{2021-01-17}]{One can't prove that God doesn't exist. \textcolor{gray}{[\ldots]} But science makes God unnecessary. \textcolor{gray}{[\ldots]} The laws of physics can explain the universe without the need for a creator.}

\paragraph{Overview}
In the following, this document tries to assess the question of whether the idea of any higher being may be beneficial, in form of e.g. ethical requirements, or if it hinders scientific development.
Section~\ref{sec:History} will start with a brief overview of the history of scientific development and the ways religious beliefs have hindered or eased them.
After this, Section~\ref{sec:Theories} will take a look at present theories on how to combine or separate religion and science.
Section~\ref{sec:Consciousness} will elaborate on the positions of some important contributors and the ways information technology changed the view on the concept of consciousness and the mind-body problem.
To sum it up,  Section~\ref{sec:Discussion} and Section~\ref{sec:Conclusion} try to assess a conclusion by using the previously discussed findings.

\paragraph{Wording}
Some words have multiple meanings depending on the context, e.g. belief.
To clarify their usage in this document they are explained here as a way of guidance.
All of the following definitions stand, if not stated explicitly otherwise:
\begin{description}
    \item[Belief:] Will be used as \say{belief in a higher power, something devine}.
    \item[God:] Will be used as an abstract name for a higher being and does not necessarily refer to the Christian god.
\end{description}
Whenever the word \emph{church} or \emph{religion} is used it will be accompanied with a specification, stating which church or religion is referred to.
\section{History}
\label{sec:History}

\begin{figure*}
    \GetAsq{timeline}
    \caption[The Timeline]{Timeline of the presented persons. Those grayed out are left as a reference for other mentions in this document~\cite{wikimedia2021}. Rounded rectangles (\tikz[baseline]{\fill[paletteA,opacity=0.35,rounded corners=1.25pt] (0,0) rectangle (0.5,.5\baselineskip)} and \tikz[baseline]{\fill[gray,opacity=0.35,rounded corners=1.25pt] (0,0) rectangle (0.5,.5\baselineskip)}) denote the full life span of the respective person.}
    \label{fig:timeline}
\end{figure*}

This section will give a brief overview of some of the important contributions in history and how they interact with the concept of a god.
As a summary, Figure~\ref{fig:timeline} shows all the people mentioned in this section and some more and puts them in a time frame.

\paragraph{Plato}
The Athenian philosopher Plato, a disciple of Socrates, lived in ancient Greece roundabout \bce{400}.
While he is probably most famous for his school of Platonism, this document will focus on his views on consciousness and the interaction between body and mind.

Plato founded the Subject-Object-Problem, which was probably processed epistemologically by Thomas Hyde in his mind-body dualism~\cite{plato360}.
From Plato's point of view, living beings are a construct of an ephemeral body and an immortal soul, whereby the soul is the life principle and the actual self of the living being at the same time.\shortfooturl{https://www.spektrum.de/lexikon/neurowissenschaft/dualismus/3052}{\formatshort{https://spektrum.de}{dualismus}}{2021-01-17}

% [TODO: descartes? $\to$ rest with aristotle?]

\paragraph{Aristotle}
The Greek polymath and philosopher Aristotle was a student of Plato and worked directly with him and later independently.
His views on dualism are quite similar, he even indulged in the theory of multiple souls, some of which die with the body, while others remain~\cite{aristotle350,hicks2015aristotle}.

In his famous work \say{Metaphysics} Aristotle writes about \say{Being}, presumably heavily influenced by Plato, his teacher.
Aristotle states metaphysics as a science that takes precedence over all other sciences and that may be characterized in three ways~\cite{aristotle350}:
\begin{itemize}
    \item \emph{Ontology}, asking about what \say{Being} is (in the highest degree).
    \item \emph{Theology}, asking about existence and the unmoved mover as the primary cause for motion in the universe.
    \item \emph{Meta-science}, dealing with evidence and first principles of thought.
\end{itemize}
Yet, the whole scope of the metaphysics-collection, dealing with the character of definitions, identity, causality, and more, exceeds the scope of this document.
The most important consensus to be taken from it is the perception of the \mbox{mind/soul} as something divine, never being able to be explained by the sciences,
while the body (the matter) is physical and examinable.

% [TODO: Metaphysics $\to$ view of god (divine world)]

\paragraph{Copernicus and Galilei} In 1543 Nicolaus Copernicus created the heliocentric model, placing the sun at the center and all other planets of the solar system orbiting around it~\cite{copernicus1965revolutionibus}.
However, the model contradicted the old Testament's geocentric worldview,\footurl[See~]{https://biblia.com/bible/esv/joshua/10/12-13}{2021-01-23} which is why it was not widely accepted for roundabout 150 years until it was finally proven by Sir Isaac Newton.

One of the most famous representatives of the heliocentric model is Galileo Galilei, who got into a dispute with the Christian Church in the early 17\textsuperscript{th} century~\cite{folsing1983galileo}.
While the Church allowed him to speak of the heliocentric system as a hypothesis (he was even encouraged by Pope Urban \Rom{8}), Galileo's work \say{Dialogo}~\cite{galilei1632dialogo} overran and earned him house arrest and a teaching ban.
It was not until November 2, 1992, that Galileo was rehabilitated by the Catholic Church.

% [TODO: Heliocentric world view $\to$ fight vs. church]

\paragraph{Charles Darwin} In his work \say{On the origin of species}~\cite{darwin1859origin}, Charles Robert Darwin published
 his theory of evolution in the year \citeyear{darwin1859origin} and was heavily criticized only one year later in a publication named \say{Essays and Reviews}~\cite{temple1860essays} mostly written by members of the Church of England.

 Although the theory of evolution gained acceptance in science rather quickly, it has been labeled a heresy by some Church officials (e.g in the aforementioned \say{Essays and Reviews}~\cite{temple1860essays}) and has left an ongoing conflict in some countries, such as the United States.
 In these countries, a not insignificant number of so-called \say{Creationists} believe in world history faithful to the bible (whereby creationism is represented in many religions~\cite{Hameed1637}) or in something called \say{Theistic Evolution}: an evolution that is compatible with religious belief and proof of gods design.
% TODO: criticue richard dawkins
% [TODO: Evolution Theory $\to$ gods work]

\paragraph{Alan Turing} As one of the most important code breakers during the Second World War, Alan Turing introduced one of the elementary computer models as early as 1936: the Turing machine~\cite{turing1936turing}.
While the discovery as such was already revolutionary, it laid the foundation for research in artificial intelligence, opening up a whole new perspective for human consciousness.
Furthermore, the Turing test (based on an idea by Alan Turing) is an example of numerous tests that attempt to get to the bottom of the peculiarity of the human mind.

Nowadays, a lot of people believe in a technological singularity, a hypothetical point in the (near future) where machines and artificial intelligence exceeds human capacities.
Amongst those are Elon Musk and Stephen Hawking, strongly believing in the capabilities of artificial intelligence.\shortfooturl{https://www.dailydot.com/irl/superintelligence-meets-religion/}{\formatshort{https://dailydot.com}{superintelligence}}{2021-01-23}

% [TODO: Turing Machine $\to$ Neural Net $\to$ consciousness $\to$ god-given? ]

% https://en.wikipedia.org/wiki/Artificial_consciousness
\section{Theories}
\label{sec:Theories}
A lot of people~-- scientists and theologians alike~-- have tried to model possible relationships between science and religion.

This section takes a close look at four of the most popular models using the names coined by John Polkinghorne~\cite{Polkinghorne1998}: \begin{inlist}
    \item the incompatibility model
    \item the independence model
    \item the dialogue model
    \item the integration model
\end{inlist}
 (Others, e.g. \citeauthor{barbour2000science}~\cite{barbour2000science} and \citeauthor{Peacocke1981}~\cite{Peacocke1981} propose similar models with different names).
Furthermore, it will take a look at the specific perspectives of some religions.

\def\lmodel#1{\includegraphics[width=1.5cm]{figures/asq_models_#1-compressed.pdf}}
\newcommand*\modelpreview[2][0.2]{\needspace{4\baselineskip}\subsection[The #2 models]{The #2 models\hfill\smash{\raisebox{-#1\baselineskip}{\lmodel{#2}}}}}
\def\shortverprefix{\thinspace\faCaretRight~}
\long\def\shortver#1{{\color{gray}
    \shortverprefix\parbox[t]{\linewidth-\widthof{\shortverprefix}}{\textit{#1}}
}}

\modelpreview{incompatibility}\label{subsec:incompatibility}
\shortver{Science and religion are fundamentally incompatible. Either there is only religion or only science.}

There is not just one incompatibility model, some of them favor scientific, others favor religious views.
Yet they all view the rational approach of science as incompatible with a divine explanation and thus represent an extreme case of the science-religion-relationship.

In this document the incompatibility models are separated into two groups: \begin{inlist}
    \item models in favor of science
    \item models in favor of religion.
\end{inlist}
All of the presented models have been or are heavily criticized from the other side, but their popularity has declined sharply since their peak in the 19\textsuperscript{th} century. Nowadays, a more nuanced view (as discussed with the other models) is generally favored~\cite{Ferngren2002, Jones2011}.

% 19 Jhd. John William Draper, Andrew Dickson White.

\paragraph{Favouring science}
Some modern scientists (e.g. the aforementioned Stephen Hawking or the still alive Richard Dawkins) support science based-models that require no religion to suffice.
Some, like Richard Dawkins or Peter William Atkins,\footurl{https://winteryknight.com/tag/peter-atkins/}{2021-01-24} are even openly hostile and say that~\cite{dawkins2006god}: \say{\quoteshorten[religion] subverts science and saps the intellect}.

All of those scientists are part of a view named \emph{scientific materialism}\shortfootarchiveurl{https://web.archive.org/web/20201128022903/https://www.sciencemeetsreligion.org/philosophy/scientific-materialism.php}{\formatshort{https://sciencemeetsreligion.org/}{materialism}}{2021-01-24} which accepts the material world as the only existing reality and denies the existence of any god or a higher world.
Another view, \emph{scientific imperialism}\shortfootarchiveurl{https://web.archive.org/web/20040903060910/http://www.empireclubfoundation.com/details.asp?SpeechID=2359&FT=yes}{\formatshort{http://empireclubfoundation.com}{2359}}{2021-01-24} is a little less dismissive and accepts the existences of supernatural experiences.
Although, they are mainly used as a gap-filler and any supernatural event is to be analyzed and explained with scientific methods sooner or later~\cite{krishna1971gopi} (similar to the view of positivism). % \footurl{https://www.wissen.de/lexikon/positivismus}{2021-01-24}


\paragraph{Favouring religion}
The already mentioned Creationists (Section~\ref{sec:History}) regard religion as the only true perspective and belief their sacred texts (like the holy bible) word-by-word~\cite{Hameed1637}.

They are part of a view called \emph{religious fundamentalism}, which is most prominent in the United States.
An a little bit less strict interpretation is named \emph{intelligent design}. It regards the world as a creation made by a divine and intelligent creator, a god.
Supported by a majority of strict believing Muslims, it rejects Darwin's theory of evolution and regards it as incompatible with the Koran~\cite{Demirci2016}.


In addition to religious fundamentalism, there is another view (which has now become rather out of date): \emph{Church authority}.
This can be found, for example, in the cases of Galileo Galilei and Charles Darwin (Section \ref{sec:History}) whose findings were subordinated to the opinion of the Vatican.


In particular, in contrast to scientific materialism, there is an idealistic perspective \emph{Idealism}, in which reality is based only on human perception and exists only as some kind of spirit.\footurl{https://www.britannica.com/topic/idealism}{2021-01-31}
See Subsection~\ref{subsec:mind-body} for another view on the matter.
% TODO: \flocomment{Go beyond: mehr regeln etc (minecraft vs Welt außen).}


% Karl Giberson: Arroganz der Wissenschaft


\modelpreview[0.3]{independence}\label{subsec:independence}
\shortver{Science and religion are two different perspectives. They complement each other, but cannot be united (in a direct way).}

Similar to the category mistake,\footurl{https://plato.stanford.edu/entries/category-mistakes/}{2021-01-24} independence models (also named coexistence models) view science and religion as two independent languages that can not be translated into each other (easily).
While the \textit{Science-Language} describes the \say{real} material world, the \textit{Religion-Language} is to describe the transcendental reality.

One of the best-known representatives of this view is Albert Einstein~\cite[p.~605\,ff.]{einstein1940science}:
\say{Science without religion is lame, religion without science is blind}.
Arnold Benz shares this view and proclaims that science and religion
differ in their definition of reality (objective measurements vs. experiences) and meet only at certain points, for example in the amazement and ethics.\shortfooturl{https://www.uzh.ch/about/portrait/awards/hc/2011/theol.html}{\formatshort{https://uzh.ch}{awards-2011}}{2021-01-24}

The independence model is a rather modern view and supported by the
National Academy of Sciences.\shortfooturl{https://www.nationalacademies.org/evolution/evolution-and-society}{\formatshort{https://nationalacademies.org}{evolution}}{2021-01-24}
Furthermore, it is backed by some religious people as well, e.g. Archbishop John Habgood calling science descriptive and religion prescriptive~\cite{habgood1964religion}.
This view is further developed by the rabbi Menachem Mendel Schneerson, who states that science, because of its arbitrary axioms, is incapable of refuting the absolute truth of the Torah.\shortfooturl{https://www.chabad.org/therebbe/letters/default_cdo/aid/66593/jewish/Torah-and-Geometry.htm}{\formatshort{https://chabad.org}{66593}}{2021-01-24} This view will be analyzed further with the dialogue model coming next.

\modelpreview[0.3]{dialogue}\label{fig:dialogue}
\shortver{Science and religion overlap in their questions. Their findings must be weighed against each other.}

As a kind of compromise, dialogue models view science and religion as two overlapping fields which use different perspectives to find common and enriched results.
Yet, in the beginning, those models were only sparsely represented in favor of the other variants.
The modern view is rooted in the works of Ian Barbour: \say{Myths, models, and paradigms: A comparative study in science and religion}~\cite{barbour1976myths}, that re-raised the interest of several groups and focuses on ethical questions.

The foundations of today's ethics can be found in many ways in religions that convey ethical values and guidelines through their texts and traditions.
Therefore, there are a lot of religious perspectives that capture the value of people and their role in creation, discussions about nuclear engineering, genetic engineering, and psychological experiments exceeding the scope of this document (cf.~\cite{barbour1993ethics,reiss2001improving}).

Beyond ethical issues, the dialogue models have some problems and often require a differentiated approach:
The \emph{Church authority}-concept (cf.~Subsection~\ref{subsec:incompatibility}) has shown severe problems when either side restricts the other one (especially if they do not even allow a dialogue).
On the other hand, there are a lot of religious Nobel Prize winners~\cite{shalev2002100} and scientists, arguing for such a dialog, fully accepting scientific views like the evolution theory.\shortfooturl{https://ncse.ngo/science-and-religion-christian-scholarship-and-theistic-science}{\formatshort{https://ncse.ngo/}{religion}}{2021-01-24}

% TODO: nochmal genauer % Deutsche wikipedia hat dafür einen eigenen Abschnitt

\modelpreview{integration}\label{subsec:integration}
\shortver{Science and religion do not contradict each other. Their statements contribute to the same truth.}

From a standpoint of complex interactions, integrations models try to acknowledge mutual influences of different areas (including science and religion).
They do not just say that scientific and religious views may coexist, they emphasize them being free of any contradictions.
Perceived inconsistencies are therefore merely the consequence of a wrong or incomplete understanding.
According to Ian G. Barbour, integration models are the \say{most promising option} (of the four models presented)~\cite[p.~2]{Barbour2002}.
% NOTE: sagt auch was zu AI Seite 83 +


There are a lot of views classified as integration models (and new ones appear all the time). As a small overview, three different views are briefly highlighted below: \begin{inlist}
    \item a scientific interpretation of the Koran
    \item the process philosophy
    \item the evolution theology.
\end{inlist}

\paragraph{Koran interpretation}
Already around the 12\textsuperscript{th} century, the theologian al-Ghazālī\footnote{With full name: Abū Hāmid Muhammad ibn Muhammad al-Ghazālī.} located all knowledge (at the time) in the Koran.
He assumed that the knowledge contained in the Koran only had to be understood and thus it strengthened his belief in its divine origin.
His teachings were continued, for example, by Jalal al-Din al-Suyuti~\cite{abdurrahman2003kecsfu} in the 15\textsuperscript{th} century, who further strengthened the point of \say{all sciences} being located in the Koran.

With the 19\textsuperscript{th} century, the perspective experienced a real boom, especially with \d{T}an\d{t}āwī Jawhari's 26 volume commentary on the Koran~\cite{jawhari1932jawahir} (although it was harshly criticized for interpreting far too freely~\cite[p.~48]{Demirci2016}).
To this day, new scientific discoveries are traced back to statements in the Koran~\cite{Demirci2016}.

\paragraph{Process philosophy}
Alfred North Witehead and later his student Charles Hartshorne developed the process philosophy (later: process theology) by redefining the concept of reality.
Instead of atoms, the reality is constructed from constant change
and god is represented through creativity and order in ever-changing situations.
With this, they explain (any) God's intervention in this world by creating order in which the emerging individuals can then develop~\cite{whitehead1957process}.


\paragraph{Evolution theology}
While the creation story in Genesis\shortfootarchiveurl{https://web.archive.org/web/20210202101513/http://www.vatican.va/archive/bible/genesis/documents/bible_genesis_en.html}{\formatshort{http://vatican.va}{genesis}}{2021-03-14} seems to contradict Charles Darwin's theory of evolution if taken literally, some integration models argue for them being contradiction-free.
{\def\mto{\(\to\)}
Therefore the sequence proclaimed by Genesis: light \mto{} plants \mto{} animals \mto{} humans, is nothing more than an abstract representation (or according to the theories: verification) of Charles Darwin's theory of evolution.
}

Other variants, such as Pierre Teilhard de Chardin's theology of evolution, consider the evolution to be far from complete, striving towards a \say{point omega} that enables the unification of science and religion (the reality of the world and the reality of God~\cite{teilhard1971christianity}), which would be al-Ghazālī's idea.


\subsection{Religious views}

\label{subsec:religious-views}Up until now, the major focus lied in particular on Christianity (e.g. with the \say{Creationists}).
Nevertheless, there are~-- of course~-- a large number of other religions, some of which deal (very) differently with the topic of science or higher beings.
Therefore, this segment will briefly explain potential differences with two other religions: \begin{inlist}
    \item Hinduism, as it is said to be the oldest religion~\cite[p.~732]{Kurien2006}
    \item Buddhism, as it does not share the same conception of a god~\cite{roloff2011buddhismus}.
\end{inlist}

However, it is difficult to talk about concepts of faith without raising any conflict: every individual may hold his or her \emph{own} concept of what belief or what a god is, and they are not meant to be attacked or generalized by this brief examination.

\paragraph{Hinduism} In contrast to Christianity, Hinduism has been more open to scientific discoveries,\footurl{https://www.hinduismnet.com/hinduism_science.htm}{2021-01-31} some texts are even said to contain references supporting or underlining multiple major scientific discoveries (e.g. Einstein's Theory of Relativity).\shortfooturl{https://www.huffpost.com/entry/hinduism-science-spirituality-intersect_b_967628?guccounter=1&guce_referrer=aHR0cHM6Ly9kdWNrZHVja2dvLmNvbS8&guce_referrer_sig=AQAAABVSJMQS16DQ3vXi1Lpt__lPrEth99U2_LY3wb8ViBQpogIFTwhl0aUf__xGGdGuW9lo_s8zAPjTU_Eq6q6FLO00ybGxYH8CI5qYXQ41IE-s1QCU8JTK6nTuOcP9zqCXc-eV0J4Rj7qZlGTJaizAz8nOo8vC6bxstv9k2restybn}{\formatshort{https://huffpost.com/}{hinduism}}{2021-02-24}
This is mostly due to the fact that a lot of scientific advancements in Indian history are strongly intertwined with their religion~\cite{mitcham2005encyclopedia}.

\paragraph{Buddhism}
Especially Buddhism and Science are considered to be compatible in an extraordinary way~\cite{yong2005buddhism}.
Buddhist concepts encourage an impartial investigation of the workings of nature and most of their schools have been open to scientific discoveries~\cite{lopez2009buddhism}.
Furthermore, Buddhist practices like meditation are studied via brain-scanning and other technologies and produce invaluable insights into psychological states.\shortfooturl{https://www.bbc.com/news/world-us-canada-12661646}{\formatshort{https://bbc.com}{meditation}}{2021-01-31}
\section{Consciousness}
\label{sec:Consciousness}

When talking about science, religion, and the ways they interact, there are many areas where they clash and raise conflicts to be settled: genetic engineering, nuclear science, psychological studies, and more.

A lot of those topics have been discussed extensively~\cite{barbour1993ethics,pfleiderer2010genethics,chernus1991nuclear}, this section aims to address the concept of consciousness and artificial consciousness in particular~\cite{buttazzo2001artificial,chella2013artificial}.
Hence, the mind-body problem which deals with the role of consciousness is presented first.
Afterwards, the section examines the current state of research in those fields. %  of artificial consciousness

\def\lmibop#1{\includegraphics[height=3\baselineskip,width=1.5cm,keepaspectratio]{figures/asq_mibop_#1-compressed.pdf}}

\newcommand*\moppreview[2]{\needspace{3\baselineskip}\paragraph[#2]{\smash{\raisebox{-2.3\baselineskip}{\lmibop{#1}}} #2}%
\parshape=4
0cm \linewidth
1.65cm \dimexpr\linewidth-1.65cm\relax
1.5cm \dimexpr\linewidth-1.5cm\relax
0cm \linewidth}

\subsection{The Mind-Body Problem}
\label{subsec:mind-body}The mind-body problem groups many theories concerning the relationship between the mind (with its thoughts and creativity) and the body (producing stimuli).
One of those theories was already mentioned in Section~\ref{sec:History} with Plato and Aristotle: Dualism.
Yet, other theories are supporting Monism, a way of viewing consciousness and mind as one and not as two different entities~-- they will be discussed as well.

The following paragraphs are accompanied by small illustrations using a \say{B} as short for \emph{Body} and a \say{M} as short for \emph{Mind}.\shortfooturl[Inspired by:~]{https://en.wikipedia.org/wiki/File:Dualism-vs-Monism.png}{\formatshort{en.wikipedia.org}{Dualism-vs-Monoism.png}}{2021-02-24}


\moppreview{dualism}{Cartesian Dualism}
Many dualistic views have been discussed by a lot of philosophers in the history of humankind (e.g. Plato and Aristotle).
They all represent roughly the same concept found in cartesian dualism, a doctrine formulated by René Descartes~\cite{Leach2017}.
Accordingly, body and mind are two different and independently existing entities that causally interact with each other.
For Descartes there is an immaterial substance (\say{res cogitans}) capable of thinking and a material substance (\say{rex extensa}) incapable of thinking but responsible for physical processes.\shortfootarchiveurl{https://web.archive.org/web/20210105192001/https://iep.utm.edu/descmind/}{https://iep.utm.edu/descmind/}{2021-01-31}


\moppreview{physicalism}{Physicalism}
One monistic view on the mind-body problem is physicalism, stating that everything that exists is physical~\cite{stoljar2010physicalism}. In consequence, the mind with all of its creativity is just the result of all the physical substance constructing the brain without anything \say{higher} at play.
In fact, physicalism goes beyond the scope of the mind-body problem,\footurl{https://plato.stanford.edu/entries/physicalism/}{2021-01-31} comparable to scientific materialism (cf.~Subsection~\ref{subsec:incompatibility})~\cite{Crane1990}.

\moppreview{idealism}{Idealism}
Another monistic view is idealism which has been mentioned along the incompatibility models favoring religion in Subsection~\ref{subsec:incompatibility}.
Idealistic perspectives consider \say{reality} as indistinguishable from human perception.
Thus, everything that \say{exists} only does so in the mind.\shortfooturl{https://www.qcc.cuny.edu/SocialSciences/ppecorino/INTRO_TEXT/Chapter\%206\%20Mind-Body/Monism-Idealism.htm}{\formatshort{https://qcc.cuny.edu}{Idealism}}{2021-02-24}
Yet, there are a lot of different variants.\footurl{https://plato.stanford.edu/entries/idealism/}{2021-02-24}

\moppreview{monism}{Neutral Monism}
There is not \say{the one} neutral monism. The term itself groups all theories which consider another fundamental nature to be responsible for the mind and the body (e.g. Ernst Mach considered this to be the combination of elements).
Baruch Spinoza and David Hume are viewed as the originators of neutral monism~\cite{rosenkrantz2010historical}, but their neutrality is sometimes criticized.\footurl[See 4.,~]{https://plato.stanford.edu/entries/neutral-monism/}{2021-02-24}

% Problem between body and mind https://en.wikipedia.org/wiki/File:Dualism-vs-Monism.png

% Done? \flocomment{Dualismus und Monismus sowie diverse weiter Betrachtungen. Hier vielleicht noch ne Grafik beziehungs\-weise wieder so wie im 3. Abschnitt?}

\subsection{The view of Information Technology}

{\columnsep=2ex\begin{wrapfigure}[11]{r}{.45\linewidth}
\vspace*{-.9\baselineskip}
\GetAsq{information_processing}
\caption{Cognitivistic information processing.}
\label{fig:info-proc-coc}
\end{wrapfigure}
While most of the presented theories focus on philosophical aspects, there are more which assess the mind-body problem.
Psychology, especially social psychology, pursues the concept of embodiment~\cite{meier2012embodiment} with a cognitivistic approach:
the mind needs the body to exist, yet all those stimuli generated by the body are processed as a whole and therefore not directly mapped to corresponding brain functions (see Figure~\ref{fig:info-proc-coc}).\par}

With the advent of information technology and neural networks, research on artificial consciousness began. If it would be possible to create consciousness solely through a programmed machine, this would not just favor physicalism but it would also undermine prominent religious views of consciousness as being something higher.

But what exactly is consciousness and what are the minimum requirements (e.g. for a machine) to be considered conscious?

\paragraph{Consciousness \& Turing} Alan Turing (Section~\ref{sec:History}) formulated his famous Turing test (originally named \say{imitation game}):\footurl{https://www1.wdr.de/wissen/technik/turing-test-100.html}{2021-03-14} a human asks questions using only a
keyboard and a monitor interfacing with two unknown partners. One of those partners is a human, while the other one is a machine. If the asking human is unable to determine who is who (machine or human), the machine should be considered to be \say{equal to the human}.

Yet, the test is heavily criticized for testing the concept of consciousness since basic heuristic principles may be enough (and may already have been enough\shortfooturl{https://www.heise.de/newsticker/meldung/Eugene-und-der-angeblich-bestandene-Turing-Test-So-einfach-nun-dann-doch-nicht-2218151.html}{\label{ftn:eugene}\formatshort{https://heise.de}{eugene}}{2021-03-14}) to fool a human into thinking that he (or she) talks to another one~\cite{john1980minds}.

\paragraph{Consciousness today} As the awareness of internal or external existence is highly subjective, there is no simple way of checking if a machine is \emph{really} self-conscious. Besides the Turing test, there is a countless number of other tests. They may all fail though, as any introspective or conscious sounding sentence can be the result of external programming or a machine which has been trained to say it.
While this might sound strange, it may be the same with humans: Solipsism\footurl{https://www.britannica.com/topic/solipsism}{2021-03-14} considers any other consciousness apart from the one of oneself to be unprovable.

Nevertheless, Baars's work~\cite{baars1993cognitive} should not go unmentioned as he lists a very convincing set of functions that have to be met to at least consider the possibility of artificial consciousness.


\paragraph{Artificial approaches}
The Turing test may have been beaten already,\footrefmark{ftn:eugene} yet there is no widely accepted self-conscious program as of today (and maybe, there never will be one~\cite[p.~231]{Meissner2020}). However, there are some promising approaches\ldots

BabyX by Soul\,Machines is their first prototype that simulates an infant. By using neural networks, BabyX can evolve during real-time face-to-face interactions in a very compelling way~\cite{Sagar2015}.

% \footurl{https://www.digitaltrends.com/cool-tech/sophia-android/}{2021-03-14}
% \shortfooturl{https://www.iflscience.com/technology/saudi-arabia-becomes-first-country-to-grant-citizenship-to-a-robot/}{\formatshort{https://iflscience.com}{sophia-citizen}}{2021-03-14}
Sophia is an adult android which has been granted citizenship in Saudi Arabia (in October 2017) and received an United Nation title just one month later.\shortfooturl{https://www.asia-pacific.undp.org/content/rbap/en/home/presscenter/pressreleases/2017/11/22/rbfsingapore.html}{\formatshort{https://asia-pacific.undp.org}{sophia-un}}{2021-03-14} While Sophia appears to be \say{alive}, such statements are heavily criticized.\shortfooturl{https://www.theverge.com/2017/11/10/16617092/sophia-the-robot-citizen-ai-hanson-robotics-ben-goertzel}{\formatshort{https://theverge.com}{sophia-critique}}{2021-03-14}

Besides those and countless other examples, some modern approaches try to use quantum computing (cf.~\cite{saxena2013}).
In \citetitle{Meissner2020}, \citeauthor{Meissner2020} gives a very well written deeper dive into the topic and presents other aspects of consciousness and difficulties in recreating them~\cite{Meissner2020}.

\section{Discussion}
\label{sec:Discussion}
\paragraph{Disclaimer} This section is heavily influenced by my opinion and does not try to convey \say{the ultimate truth}. As already mentioned in Subsection~\ref{subsec:religious-views}, every individual may hold his or her \emph{own} concept of what belief or a god is. I neither intend nor want to attack anyone who has a different attitude or even strongly disagrees with my take on the topic.

In four steps, I will try to give a brief overview of my thoughts.
% Starting with some thoughts about the core problem I will continue with the (luckily sparse) authoritarian views and then focus on the relationship that I desire (and why).
% I will start with some words about the (luckily sparse) authoritarian views, then write about the relationship desired by me

\subsection{The Unknown}\label{subsec:the-unknown}
No matter how long I have thought about the relationship between science and religion and discussed it with others, it all reduces to one fact: we do not know. Science is (at least as of yet) unable to explain everything and no religion as of date was able to objectively prove the existence of anything supernatural.

Without any ultimate truth, the debate will probably be everlasting. However, scientific research has undoubtedly produced a lot of valuable insights and a lot of once-unthinkable things are mundane today.
And while I am certainly in favor of scientific research (since I study information technology), the scientifically graspable just might be a snow globe.

\subsection{The Snow Globe}\label{subsec:snow-globe}
The \say{Snow Globe} is a perspective that I have created during countless walks and in discussions with many fellow students.
It is probably most easily explained by the analogy from which the model arouse: the game Minecraft.\footnote{Although this works with basically any sandbox game, I have chosen to stay with the game the idea originated from.}

\paragraph{The Origin} Minecraft is a sandbox game created by Markus Persson (\say{Notch}) where the player(s) can interact with a world consisting of blocks. While the basic rules of the game are fairly simple, they offer a lot of freedom, allowing to build functional computers and flying machines within the game. Still, even today new techniques are discovered (with methods comparable to modern science) and used effectively.\footnote{Of course, some of those discoveries merely stem from the fact that the game still receives updates, yet a lot of them hold for older versions.}

While \emph{we} know that Minecraft was programmed and that we do exist \say{outside of the game}, taking the perspective of an in-game character might represent the same situation as we are in, in our \say{reality}. I will coin those two states as \emph{out-game} (our reality) and \emph{in-game} (the \say{Minecraft reality}).
No matter how much research we would invest in-game, we would not know anything about the out-game world.

This perspective differs from that of the \say{The Matrix}-Trilogy\footurl{https://www.imdb.com/title/tt0133093/}{2021-03-17} because the in-game state is not the same as the  out-game state in terms of its rules.
It further differs from Abbott's \say{Flatland}~\cite{abbott1987flatland} because the out-game state \emph{created} the in-game state (furthermore, in Abbot's story they do casually interact with each other).

\paragraph{The Snow Globe}
The name \say{Snow Globe} merely originated from the fact that the interaction between out- and in-game could very well be uni-directional.
The inhabitants of snow-globe-world might see the snowfall, up and down, left and right, they might be able to discover \textit{a} gravity (which might turn its direction with the snowfall),~\ldots\ yet they would be incapable of predicting the way the snow will fall as the out-game human might shake the snow globe and be interrupted by a sneeze/situational factors incomprehensible for the snow globe inhabitants.


\paragraph{Classification}
I have created this model before I knew about all of the theories presented in Section~\ref{sec:Theories}. Mapping religion as the belief of the snow globe inhabitants in \say{us} or at least something higher than snow-globe-world, this model could be categorized as an independence (Subsection~\ref{subsec:independence}) or an integration (Subsection~\ref{subsec:integration}) model as the out-game rules influence the in-game ones (e.g. gravity).
I will reinforce this model in Subsection~\ref{subsec:golden-middle}.

\subsection{The Extremists}
Any intention may rot when the desire to enforce it blossoms in a one-dimensional and extremist perspective.
And while utopia (by definition) sounds great, everyone may perceive it differently\ldots
\blockquote[probably Immanuel Kant]{Die Freiheit des Einzelnen endet dort, wo die Freiheit des Anderen beginnt.\\\textcolor{gray}{One's freedom ends where the freedom of another begins.}}
Therefore, I do not like views such as the Church authority (see Subsection~\ref{subsec:incompatibility}) enforcing their truth as the only one, verifiably suppressing advancements and destroying existing knowledge.

Of course, this is neither limited to the relationship between science and religion,\footnote{Yet, scientific advancements are as much a part of modern societies as religions have been just a few centuries ago.} nor is this limited to oppression from a religious side. As already mentioned in Subsection~\ref{subsec:the-unknown}: we do not know.


\subsection{The Golden middle}
\label{subsec:golden-middle}
For myself, I ruled out the incompatibility model. Not just for the reasons mentioned before but for the main reason that it seems ignorant to ignore a perspective (no matter which side you are on) that you can not definitely \emph{prove} to be false.
With the other models, it is a little bit more difficult.

I do not believe in the dialogue model. While some kind of dialogue is definitely of use in the most prominent topic of ethics, I do not think that this holds for any other topic. The religious \say{discoveries} and phenomena are different from scientific ones. This might be due to the current lack of understanding in the human psyche, but if so, I think this does not support the dialogue model as it would either mean \begin{orlist}
    \item a contradict free physical description of religious belief (supporting the integration model)
    \item another hint for the inexplicability of those events by scientific standards (supporting the independence model).
\end{orlist}

With this argument and the \say{Snow Globe}, already discussed in Subsection~\ref{subsec:snow-globe}, I tend to support the independence or integration model: even if we can create something that appears to be artificial consciousness, consciousness as we know it might just be another artifact of \say{our} reality.

I tend to the independence side: currently, it just seems far more plausible that we are not able to see out of our snow-globe-reality.
While ethics were in fact primarily shaped by religion (at least in the beginning), I would argue for them being merely a side effect of human evolution and therefore not solely part of the religious side.
\section{Conclusion}
\label{sec:Conclusion}
\flocomment{Naja, das muss ich auch noch machen xD}

Umfrage Wissenschaftler \(\to\) integrationsmodell \(\to\) limits
% https://en.wikipedia.org/wiki/Relationship_between_religion_and_science#Surveys_on_scientists_and_the_general_public_on_science_and_religion

bezüglich der Modelle:
Gary Ferngren, has stated: "Although popular images of controversy continue to exemplify the supposed hostility of Christianity to new scientific theories, studies have shown that Christianity has often nurtured and encouraged scientific endeavour, while at other times the two have co-existed without either tension or attempts at harmonization. If Galileo and the Scopes trial come to mind as examples of conflict, they were the exceptions rather than the rule."[64]

=> indepenent bis wir irengwannmal das große ganze verstehen, falls wir das jemals tun?
% ich glaube nicht, dass eine höhere Macht existiert, aber ich glaube daran, dass der glaube an die Existenz einer höheren Macht psychologisch wichtig ist und viel liefern kann. (Ich)

% kritisiere Glaube nicht.

\appendix
\defbibnote{generic@asq@prenote}{All links cited in the footnotes are not listed again. All created images (if not stated otherwise) have been created by the author (\typesetAuthor) using Ti\textit{k}Z.}
\printbibliography[prenote=generic@asq@prenote]

\end{document}