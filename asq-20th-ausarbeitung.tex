\errorcontextlines 999999
\documentclass[userip,english,cpalette=Crimson]{lecture-fancy-paper}
\PassOptionsToPackage{threshold=1}{csquotes}
\usepackage[%
    sopra-listings={cpalette,numinpar},%
    sopra-tables={cpalette},%
    sopra-models,%
    lecture,%
    lecture-bibliography={biber,style=numeric},%
    lithie-boxes={germanenv,autostyle},%
    lithie-task-boxes={cpalette},%
    lecture-links={patchurl}%
]{lithie-util}

\input{color_definition.src}
\pagecolor{white!97!lightgray!99!paletteA}

\usepackage[style=mmddyyyy,yearmonthsep={/},dayyearsep={/},monthdaysep={/}]{datetime2}

\LoadBoxStyle{limerence}

\addbibresource{bibliography-collection/lithie.bib}

\makeatletter
\def\input@path{{source/}{figures/}}
\makeatother

\loadglsentries{glossary}

\usepackage[imgprefix={figures/asq_},command={Asq}]{lecture-personal}

\title{Limitations of science}%
\subtitle{The inexplicable, god, and its development in science}
\def\titleOfASQ{20\textsuperscript{th}~Century Thinking about Science}
\brief{\disablehyper\titleOfASQ~(\uc{asq})\\*\typesetAuthor, \lectureGetRegister{date}}
\addAuthor{Florian Sihler (florian.sihler@uni-ulm.de)}
\date\DTMtoday
\subject{ASQ, Elaboration}
\supervisor{Dr. Hans-Peter Eckle}
\authority{\textcopyright~2021 Uni Ulm}
\copyright{\href{https://creativecommons.org/licenses/by-nc-sa/2.0/}{CC BY-NC-SA 2.0} \faCreativeCommons\ \faCreativeCommonsBy\ \faCreativeCommonsNcEu\ \faCreativeCommonsSa}

\usepackage{blindtext}

\newcommand*\footurl[3][]{\footnote{#1\url{#2}\quad(\DTMdate{#3})}}
\def\UrlBreaks{\do\/\do-}
\newcommand*\shortfooturl[4][]{\footnote{#1\href{#2}{#3}\quad(\DTMdate{#4})}}
\newcommand*\shortfootarchiveurl[4][]{\footnote{#1\href{#2}{#3}\quad(archived, \DTMdate{#4})}}
% \;\smash{\raisebox{0.3\baselineskip}{\scalebox{0.35}{\faUniversity}}}

\usepackage{pbox}
\def\flocomment#1{\fcolorbox{paletteA}{shadeA}{\pbox{\linewidth}{\sffamily\textbf{\paletteA{Flo:}}~#1}}}

\def\formatshort#1#2{#1~\smash{\raisebox{0.25\baselineskip}{\resizebox{!}{0.25\baselineskip}{\pgfornament{14}}}}~#2}

\newcommand*\bce[1]{#1~BCE} % keep consistent with bibrefs
\SetBlockThreshold{0}

\usepackage{newunicodechar}
\newunicodechar{Ṭ}{\d{T}}
\newunicodechar{ṭ}{\d{t}}
\newunicodechar{ṣ}{\d{s}}
\newunicodechar{Ḥ}{\d{H}}
\newunicodechar{ī}{\=\i}
\newunicodechar{ā}{\=a}

\begin{document}

\maketitle
% \lecturedivider

\begin{abstract}
    Since the dawn of mankind, humans are trying to understand how the world works.
    These explanations, especially at the beginning, were dominated by a divine view of the world, which was supplemented by more and more science-based knowledge over time.

    This elaboration gives a brief overview of the history of the most important changes in the scientific worldview and their limitations.
    Furthermore, the meaning of those limitations will be discussed, using the relationship to a god of some of the important contributors.

    It turns out that an (even imagined) limit in the perceivable worldview is not beneficial to scientific research. In addition, the integration of religion into science even seems to be beneficial.
    % and in no way can science ever fully replace religion
\end{abstract}

This work was created as an elaboration for the \uc{asq}: \say{\titleOfASQ} by \lectureGetRegister{supervisor} in the winter-semester 2020/2021 at the University of Ulm. The \LaTeXe-Documentclass was custom made for this document by the author.

\section{Introduction}

When talking about the origins of science, Aristotle is one of the most important figures, shaping the basic cycle of induction \& deduction and laying the groundwork for western science.\footurl{https://www.britannica.com/biography/Aristotle}{2021-16-01}
While Aristotle tried to fathom the world and its causal workings, he still believed in a god/in something divine.
His famous work \say{Metaphysics} establishes the science of the divine (theology) as one of the three important pillars, next to the ontology and the science of general principles~\cite{aristotle350}.
However, at such an early stage it may seem logical to use belief  (as belief in something higher, something divine) to fill in the blanks that science is unable to explain (yet).

These days we are much more enlightened and informed about how the world works and how we, as humans, work, questioning the need for belief.
Physics, biology, and chemistry, together with mathematics and especially with the information-technology revolution, opened up areas that were unthinkable at the time.
Some of those discoveries, such as Galileo's heliocentric worldview or Darwin's theory of evolution, can even be equated with blasphemy from the perspective of the (Christian) church, at least at the time they have been discovered.

Still, a lot of modern scientists, like Sir Isaac Newton~\cite[p. 315]{westfall1983} or Jérôme Lejeune,\footurl{https://lejeunefoundation.org/jerome-lejeune/}{2021-16-01} do believe in a god (or a higher power in general), others like Richard Feynman~\cite{feynman2001,brian2001} and Steven Hawking\footurl{https://time.com/5199149/stephen-hawking-death-god-atheist/}{2021-16-01} do not.
To quote the latter one:
\blockquote[Stephen Hawking\footurl{https://abcnews.go.com/GMA/stephen-hawking-science-makes-god-unnecessary/story?id=11571150}{2021-17-01}]{One can't prove that God doesn't exist. \textcolor{gray}{[\ldots]} But science makes God unnecessary. \textcolor{gray}{[\ldots]} The laws of physics can explain the universe without the need for a creator.}

\paragraph{Overview}
In the following, this document tries to assess the question if the conception of any higher being may be beneficial, in form of e.g. ethical requirements, or if it hinders scientific development.
Section~\ref{sec:History} will start with a brief overview of the history of scientific development and the ways religious beliefs have hindered or eased them.
After this, Section~\ref{sec:Theories} will take a look at present theories on how to combine or separate religion and science.
Section~\ref{sec:Consciousness} will elaborate on the positions of some important contributors and the ways information technology changed the view on the concept of consciousness.
To sum it up, Section~\ref{sec:Conclusion} tries to assess a conclusion by using the previously discussed findings.

\paragraph{Wording}
Some words have multiple meanings depending on the context, e.g. belief.
To clarify their usage in this document they are explained here as a way of guidance.
All of the following definitions stand, if not stated explicitly otherwise:
\begin{description}
    \item[Belief:] Will be used as \say{belief in a higher power, something devine}.
    \item[God:] Will be used as an abstract name for a higher being and does not necessarily refer to the Christian god.
\end{description}
Whenever the word \emph{church} or \emph{religion} is used it will be accompanied with a specification, stating which church or religion is referred to.
\section{History}
\label{sec:History}

\begin{figure*}
    \GetAsq{timeline}
    \caption{A timeline}
    \label{fig:timeline}
\end{figure*}

Hier soll ein Mega Bild mit einer ultra coolen Zeitleiste landen eins eins elf!
\section{Theories}
\label{sec:Theories}
\section{Consciousness}
\label{sec:Consciousness}


\subsection{The Mind-Body Problem}
% Problem between body and mind https://en.wikipedia.org/wiki/File:Dualism-vs-Monism.png


\subsection{Information technology}

\section{Conclusion}
\label{sec:Conclusion}


Umfrage Wissenschaftler \(\to\) integrationsmodell \(\to\) limits
% https://en.wikipedia.org/wiki/Relationship_between_religion_and_science#Surveys_on_scientists_and_the_general_public_on_science_and_religion

bezüglich der Modelle:
Gary Ferngren, has stated: "Although popular images of controversy continue to exemplify the supposed hostility of Christianity to new scientific theories, studies have shown that Christianity has often nurtured and encouraged scientific endeavour, while at other times the two have co-existed without either tension or attempts at harmonization. If Galileo and the Scopes trial come to mind as examples of conflict, they were the exceptions rather than the rule."[64]


\appendix
\defbibnote{generic@asq@prenote}{All links cited in the footnotes are not listed again. All created images (if not stated otherwise) have been created by the author (\typesetAuthor) using Ti\textit{k}Z.}
\printbibliography[prenote=generic@asq@prenote]

\end{document}