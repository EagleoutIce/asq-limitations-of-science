\errorcontextlines 999999
\documentclass[userip,english,cpalette=Crimson]{lecture-fancy-paper}
\PassOptionsToPackage{threshold=1}{csquotes}
\PassOptionsToPackage{style=mmddyyyy,yearmonthsep={/},dayyearsep={/},monthdaysep={/}}{datetime2}
\usepackage[%
    sopra-listings={cpalette,numinpar},%
    sopra-tables={cpalette},%
    sopra-models,lecture,%
    lecture-bibliography={biber,style=numeric},%
    lithie-boxes={germanenv,autostyle},%
    lithie-task-boxes={cpalette},%
    lecture-links={patchurl}%
]{lithie-util}
\usepackage{ccicons}

\input{figures/color_definition.src}
\pagecolor{white!97!lightgray!99!paletteA}

\LoadBoxStyle{limerence}

\addbibresource{bibliography-collection/lithie.bib}

\makeatletter
\def\input@path{{source/}{figures/}}
\makeatother

\loadglsentries{glossary}
\usepackage{wrapfig,lipsum}
\usepackage[imgprefix={figures/asq_},command={Asq}]{lecture-personal}

\title{Limitations of Science}
\subtitle{The inexplicable, god, and its development in science}
\def\titleOfASQ{20\textsuperscript{th}~Century Thinking about Science}
\brief{\disablehyper\titleOfASQ~(\uc{asq})\\*\typesetAuthor, \lectureGetRegister{date}}
\addAuthor{Florian Sihler (florian.sihler@uni-ulm.de)}
\date\DTMtoday
\subject{ASQ, Elaboration}
\supervisor{Dr. Hans-Peter Eckle}
\authority{Ulm University, 2021}
\copyright{\href{https://creativecommons.org/licenses/by-nc-sa/2.0/}{CC BY-NC-SA 2.0}~~\ccLogo\ \ccAttribution\ \ccNonCommercialEU\ \ccShareAlike}

\newcommand*\footurl[3][]{\footnote{#1\url{#2}\quad(\DTMdate{#3})}}
\def\UrlBreaks{\do\/\do-}
\newcommand*\shortfooturl[4][]{\footnote{#1\href{#2}{#3}\quad(\DTMdate{#4})}}
\newcommand*\shortfootarchiveurl[4][]{\footnote{#1\href{#2}{#3}\quad(archived, \DTMdate{#4})}}
\newcommand*\footrefmark[1]{\textsuperscript{\disablehyper\hyperref[#1]{\thinspace\textcolor{gray}{see} \ref*{#1}}}}

% \;\smash{\raisebox{0.3\baselineskip}{\scalebox{0.35}{\faUniversity}}}

\usepackage{pbox}
\def\flocomment#1{\fcolorbox{paletteA}{shadeA}{\pbox{\linewidth}{\sffamily\textbf{\paletteA{Flo:}}~#1}}}

\def\formatshort#1#2{#1~\smash{\raisebox{.25\baselineskip}{\resizebox{!}{.25\baselineskip}{\pgfornament{14}}}}~#2}

\newcommand*\bce[1]{#1~BCE} % keep consistent with bibrefs
\SetBlockThreshold{0}

\usepackage{newunicodechar}
\newunicodechar{Ṭ}{\d{T}}
\newunicodechar{ṭ}{\d{t}}
\newunicodechar{ṣ}{\d{s}}
\newunicodechar{Ḥ}{\d{H}}
\newunicodechar{ī}{\=\i}
\newunicodechar{ā}{\=a}

\renewcommand*{\bibfont}{\normalfont\small}

\newcommand*\quoteshorten[1][$\mathellipsis$]{\textcolor{gray}{[}#1\textcolor{gray}{]}}

\begin{document}

\maketitle
% \lecturedivider

\begin{abstract}
    Since the dawn of mankind, humans are trying to understand how the world works.
    Their explanations, especially at the beginning, were dominated by a divine view of the world, which was supplemented by more and more science-based knowledge over time.

    This elaboration gives a brief overview of the history of the most important changes in the scientific worldview and their limitations, focusing on western cultures.
    Furthermore, the meaning of those limitations will be discussed, using the relationship to a god of some important scientific-contributors.

    It turns out that an (even imagined) limit in the perceivable worldview is not beneficial to scientific research. The union of religion and science even seems to be favorable.
    % and in no way can science ever fully replace religion
\end{abstract}

This work was created as an elaboration for the \uc{asq}: \say{\titleOfASQ} by \lectureGetRegister{supervisor} in the winter-semester 2020/2021 at the University of Ulm. The \LaTeXe-Documentclass was custom-made for this document by the author.

\section{Introduction}

When talking about the origins of science, Aristotle is one of the most important figures, shaping the basic cycle of induction \& deduction and laying the groundwork for western science.\footurl{https://www.britannica.com/biography/Aristotle}{2021-16-01}
While Aristotle tried to fathom the world and its causal workings, he still believed in a god/in something divine.
His famous work \say{Metaphysics} establishes the science of the divine (theology) as one of the three important pillars, next to the ontology and the science of general principles~\cite{aristotle350}.
However, at such an early stage it may seem logical to use belief  (as belief in something higher, something divine) to fill in the blanks that science is unable to explain (yet).

These days we are much more enlightened and informed about how the world works and how we, as humans, work, questioning the need for belief.
Physics, biology, and chemistry, together with mathematics and especially with the information-technology revolution, opened up areas that were unthinkable at the time.
Some of those discoveries, such as Galileo's heliocentric worldview or Darwin's theory of evolution, can even be equated with blasphemy from the perspective of the (Christian) church, at least at the time they have been discovered.

Still, a lot of modern scientists, like Sir Isaac Newton~\cite[p. 315]{westfall1983} or Jérôme Lejeune,\footurl{https://lejeunefoundation.org/jerome-lejeune/}{2021-16-01} do believe in a god (or a higher power in general), others like Richard Feynman~\cite{feynman2001,brian2001} and Steven Hawking\footurl{https://time.com/5199149/stephen-hawking-death-god-atheist/}{2021-16-01} do not.
To quote the latter one:
\blockquote[Stephen Hawking\footurl{https://abcnews.go.com/GMA/stephen-hawking-science-makes-god-unnecessary/story?id=11571150}{2021-17-01}]{One can't prove that God doesn't exist. \textcolor{gray}{[\ldots]} But science makes God unnecessary. \textcolor{gray}{[\ldots]} The laws of physics can explain the universe without the need for a creator.}

\paragraph{Overview}
In the following, this document tries to assess the question if the conception of any higher being may be beneficial, in form of e.g. ethical requirements, or if it hinders scientific development.
Section~\ref{sec:History} will start with a brief overview of the history of scientific development and the ways religious beliefs have hindered or eased them.
After this, Section~\ref{sec:Theories} will take a look at present theories on how to combine or separate religion and science.
Section~\ref{sec:Consciousness} will elaborate on the positions of some important contributors and the ways information technology changed the view on the concept of consciousness.
To sum it up, Section~\ref{sec:Conclusion} tries to assess a conclusion by using the previously discussed findings.

\paragraph{Wording}
Some words have multiple meanings depending on the context, e.g. belief.
To clarify their usage in this document they are explained here as a way of guidance.
All of the following definitions stand, if not stated explicitly otherwise:
\begin{description}
    \item[Belief:] Will be used as \say{belief in a higher power, something devine}.
    \item[God:] Will be used as an abstract name for a higher being and does not necessarily refer to the Christian god.
\end{description}
Whenever the word \emph{church} or \emph{religion} is used it will be accompanied with a specification, stating which church or religion is referred to.
\section{History}
\label{sec:History}

\begin{figure*}
    \GetAsq{timeline}
    \caption{A timeline}
    \label{fig:timeline}
\end{figure*}

Hier soll ein Mega Bild mit einer ultra coolen Zeitleiste landen eins eins elf!
\section{Theories}
\label{sec:Theories}
\section{Consciousness}
\label{sec:Consciousness}


\subsection{The Mind-Body Problem}
% Problem between body and mind https://en.wikipedia.org/wiki/File:Dualism-vs-Monism.png


\subsection{Information technology}

\section{Discussion}
\label{sec:Discussion}
\paragraph{Disclaimer} This section is heavily influenced by my opinion and does not try to convey \say{the ultimate truth}. As already mentioned in Subsection~\ref{subsec:religious-views}, every individual may hold his or her \emph{own} concept of what belief or a god is. I neither intend nor want to attack anyone who has a different attitude or even strongly disagrees with my take on the topic.

In four steps, I will try to give a brief overview of my thoughts.
% Starting with some thoughts about the core problem I will continue with the (luckily sparse) authoritarian views and then focus on the relationship that I desire (and why).
% I will start with some words about the (luckily sparse) authoritarian views, then write about the relationship desired by me

\subsection{The Unknown}\label{subsec:the-unknown}
No matter how long I have thought about the relationship between science and religion and discussed it with others, it all reduces to one fact: we do not know. Science is (at least as of yet) unable to explain everything and no religion as of date was able to objectively prove the existence of anything supernatural.

Without any ultimate truth, the debate will probably be everlasting. However, scientific research has undoubtedly produced a lot of valuable insights and a lot of once-unthinkable things are mundane today.
And while I am certainly in favor of scientific research (since I study information technology), the scientifically graspable just might be a snow globe.

\subsection{The Snow Globe}\label{subsec:snow-globe}
The \say{Snow Globe} is a perspective that I have created during countless walks and in discussions with many fellow students.
It is probably most easily explained by the analogy from which the model arouse: the game Minecraft.\footnote{Although this works with basically any sandbox game, I have chosen to stay with the game the idea originated from.}

\paragraph{The Origin} Minecraft is a sandbox game created by Markus Persson (\say{Notch}) where the player(s) can interact with a world consisting of blocks. While the basic rules of the game are fairly simple, they offer a lot of freedom, allowing to build functional computers and flying machines within the game. Still, even today new techniques are discovered (with methods comparable to modern science) and used effectively.\footnote{Of course, some of those discoveries merely stem from the fact that the game still receives updates, yet a lot of them hold for older versions.}

While \emph{we} know that Minecraft was programmed and that we do exist \say{outside of the game}, taking the perspective of an in-game character might represent the same situation as we are in, in our \say{reality}. I will coin those two states as \emph{out-game} (our reality) and \emph{in-game} (the \say{Minecraft reality}).
No matter how much research we would invest in-game, we would not know anything about the out-game world.

This perspective differs from that of the \say{The Matrix}-Trilogy\footurl{https://www.imdb.com/title/tt0133093/}{2021-03-17} because the in-game state is not the same as the  out-game state in terms of its rules.
It further differs from Abbott's \say{Flatland}~\cite{abbott1987flatland} because the out-game state \emph{created} the in-game state (furthermore, in Abbot's story they do casually interact with each other).

\paragraph{The Snow Globe}
The name \say{Snow Globe} merely originated from the fact that the interaction between out- and in-game could very well be uni-directional.
The inhabitants of snow-globe-world might see the snowfall, up and down, left and right, they might be able to discover \textit{a} gravity (which might turn its direction with the snowfall),~\ldots\ yet they would be incapable of predicting the way the snow will fall as the out-game human might shake the snow globe and be interrupted by a sneeze/situational factors incomprehensible for the snow globe inhabitants.


\paragraph{Classification}
I have created this model before I knew about all of the theories presented in Section~\ref{sec:Theories}. Mapping religion as the belief of the snow globe inhabitants in \say{us} or at least something higher than snow-globe-world, this model could be categorized as an independence (Subsection~\ref{subsec:independence}) or an integration (Subsection~\ref{subsec:integration}) model as the out-game rules influence the in-game ones (e.g. gravity).
I will reinforce this model in Subsection~\ref{subsec:golden-middle}.

\subsection{The Extremists}
Any intention may rot when the desire to enforce it blossoms in a one-dimensional and extremist perspective.
And while utopia (by definition) sounds great, everyone may perceive it differently\ldots
\blockquote[probably Immanuel Kant]{Die Freiheit des Einzelnen endet dort, wo die Freiheit des Anderen beginnt.\\\textcolor{gray}{One's freedom ends where the freedom of another begins.}}
Therefore, I do not like views such as the Church authority (see Subsection~\ref{subsec:incompatibility}) enforcing their truth as the only one, verifiably suppressing advancements and destroying existing knowledge.

Of course, this is neither limited to the relationship between science and religion,\footnote{Yet, scientific advancements are as much a part of modern societies as religions have been just a few centuries ago.} nor is this limited to oppression from a religious side. As already mentioned in Subsection~\ref{subsec:the-unknown}: we do not know.


\subsection{The Golden middle}
\label{subsec:golden-middle}
For myself, I ruled out the incompatibility model. Not just for the reasons mentioned before but for the main reason that it seems ignorant to ignore a perspective (no matter which side you are on) that you can not definitely \emph{prove} to be false.
With the other models, it is a little bit more difficult.

I do not believe in the dialogue model. While some kind of dialogue is definitely of use in the most prominent topic of ethics, I do not think that this holds for any other topic. The religious \say{discoveries} and phenomena are different from scientific ones. This might be due to the current lack of understanding in the human psyche, but if so, I think this does not support the dialogue model as it would either mean \begin{orlist}
    \item a contradict free physical description of religious belief (supporting the integration model)
    \item another hint for the inexplicability of those events by scientific standards (supporting the independence model).
\end{orlist}

With this argument and the \say{Snow Globe}, already discussed in Subsection~\ref{subsec:snow-globe}, I tend to support the independence or integration model: even if we can create something that appears to be artificial consciousness, consciousness as we know it might just be another artifact of \say{our} reality.

I tend to the independence side: currently, it just seems far more plausible that we are not able to see out of our snow-globe-reality.
While ethics were in fact primarily shaped by religion (at least in the beginning), I would argue for them being merely a side effect of human evolution and therefore not solely part of the religious side.
\section{Conclusion}
\label{sec:Conclusion}

With Subsection~\ref{subsec:the-unknown} it may seem blunt to end with \say{we do not know, but I think\ldots} yet this is (at least in my opinion) exactly what makes this topic so interesting.
While it should be clear that one side restricting the other is not beneficial, different approaches like Levine's \say{explanatory gap}\shortfootarchiveurl{https://web.archive.org/web/20060721022750/http://www-lehre.inf.uos.de/~dbauer/stud/pom/PoM_levine.html}{\formatshort{http://www-lehre.inf.uos.de}{Levine}}{2021-03-14} show, that there are way more possible perspectives on this topic.

Furthermore, we may know, sometime in the future. The advancements in information technology make a \say{The Matrix}-scenario more and more plausible.
With artificial consciousness, we may be able to at least get further insights and thus further clues about the interplay of science and religion.

Yet, if we really do live in a snow globe will only ever be possible to discover if we do not create our own. Ethics and the resulting boundaries are important, there is no question about that (and the ethic-discussion exceeds the scope of this document).
Nevertheless, if we do not search for boundaries, we will never find them~-- and currently, we have not.

% Levine \say{Erklärungslücke}

% Umfrage Wissenschaftler \(\to\) integrationsmodell \(\to\) limits
% % https://en.wikipedia.org/wiki/Relationship_between_religion_and_science#Surveys_on_scientists_and_the_general_public_on_science_and_religion

% bezüglich der Modelle:
% Gary Ferngren, has stated: "Although popular images of controversy continue to exemplify the supposed hostility of Christianity to new scientific theories, studies have shown that Christianity has often nurtured and encouraged scientific endeavour, while at other times the two have co-existed without either tension or attempts at harmonization. If Galileo and the Scopes trial come to mind as examples of conflict, they were the exceptions rather than the rule."[64]

% => indepenent bis wir irengwannmal das große ganze verstehen, falls wir das jemals tun?
% % ich glaube nicht, dass eine höhere Macht existiert, aber ich glaube daran, dass der glaube an die Existenz einer höheren Macht psychologisch wichtig ist und viel liefern kann. (Ich)

% % kritisiere Glaube nicht.

\appendix
\defbibnote{generic@asq@prenote}{All links cited in the footnotes are not listed again. All created images (if not stated otherwise) have been created by the author (\typesetAuthor) using Ti\textit{k}Z.}
\printbibliography[prenote=generic@asq@prenote]

\end{document}