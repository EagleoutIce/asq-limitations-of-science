\section{Introduction}

When talking about the origins of science, Aristotle is one of the most important figures, shaping the basic cycle of induction \& deduction and laying the groundwork for western science.\footurl{https://www.britannica.com/biography/Aristotle}{2021-01-16}
While Aristotle tried to fathom the world and its causal workings, he still believed in a god/in something divine.
His famous work \say{Metaphysics} establishes the science of the divine (theology) as one of the three important pillars, next to the ontology and the science of general principles~\cite{aristotle350}.
However, at such an early stage it may seem logical to use belief  (as belief in something higher, something divine) to fill in the blanks that science is unable to explain (yet).

These days we are much more enlightened and informed about how the world works and how we, as humans, work, questioning the need for belief.
Physics, biology, and chemistry, together with mathematics and especially with the information-technology revolution, opened up areas that were unthinkable at the time.
Some of those discoveries, such as Galileo's heliocentric worldview or Darwin's theory of evolution, can even be equated with blasphemy from the perspective of the (Christian) church, at least at the time they have been discovered.

Still, a lot of modern scientists, like Sir Isaac Newton~\cite[p. 315]{westfall1983} or Jérôme Lejeune,\footurl{https://lejeunefoundation.org/jerome-lejeune/}{2021-16-01} do believe in a god (or a higher power in general), others like Richard Feynman~\cite{feynman2001,brian2001} and Steven Hawking\shortfooturl{https://time.com/5199149/stephen-hawking-death-god-atheist/}{\formatshort{https://time.com}{5199149}}{2021-01-16} do not.
To quote the latter, the following quote expresses Hawking's opinion quite well:
\blockquote[Stephen Hawking\shortfooturl{https://abcnews.go.com/GMA/stephen-hawking-science-makes-god-unnecessary/story?id=11571150}{\formatshort{https://abcnews.go.com}{11571150}}{2021-01-17}]{One can't prove that God doesn't exist. \textcolor{gray}{[\ldots]} But science makes God unnecessary. \textcolor{gray}{[\ldots]} The laws of physics can explain the universe without the need for a creator.}

\paragraph{Overview}
In the following, this document tries to assess the question of whether the idea of any higher being may be beneficial, in form of e.g. ethical requirements, or if it hinders scientific development.
Section~\ref{sec:History} will start with a brief overview of the history of scientific development and the ways religious beliefs have hindered or eased them.
After this, Section~\ref{sec:Theories} will take a look at present theories on how to combine or separate religion and science.
Section~\ref{sec:Consciousness} will elaborate on the positions of some important contributors and the ways information technology changed the view on the concept of consciousness and the mind-body problem.
To sum it up,  Section~\ref{sec:Discussion} and Section~\ref{sec:Conclusion} try to assess a conclusion by using the previously discussed findings.

\paragraph{Wording}
Some words have multiple meanings depending on the context, e.g. belief.
To clarify their usage in this document they are explained here as a way of guidance.
All of the following definitions stand, if not stated explicitly otherwise:
\begin{description}
    \item[Belief:] Will be used as \say{belief in a higher power, something devine}.
    \item[God:] Will be used as an abstract name for a higher being and does not necessarily refer to the Christian god.
\end{description}
Whenever the word \emph{church} or \emph{religion} is used it will be accompanied with a specification, stating which church or religion is referred to.