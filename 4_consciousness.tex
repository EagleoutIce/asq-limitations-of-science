\section{Consciousness}
\label{sec:Consciousness}

When talking are about science and religion and the ways they interact, there are many areas where they clash and raise conflicts to be settled: genetic engineering, nuclear science, psychological studies, and more.

A lot of those topics have been discussed extensively~\cite{barbour1993ethics,pfleiderer2010genethics,chernus1991nuclear}, but this section aims to address the concept of consciousness and artificial consciousness in particular~\cite{buttazzo2001artificial,chella2013artificial}.
Hence, the mind-body problem which deals with the role of consciousness is examined first.
Afterward, the section deals with the current state of research in the field of artificial consciousness.

\def\lmibop#1{\includegraphics[height=3\baselineskip,width=1.5cm,keepaspectratio]{figures/asq_mibop_#1-compressed.pdf}}

\newcommand*\moppreview[2]{\needspace{3\baselineskip}\paragraph[#2]{\smash{\raisebox{-2.3\baselineskip}{\lmibop{#1}}} #2}%
\parshape=4
0cm \linewidth
1.65cm \dimexpr\linewidth-1.65cm\relax
1.5cm \dimexpr\linewidth-1.5cm\relax
0cm \linewidth}

\subsection{The Mind-Body Problem}
\label{subsec:mind-body}The mind-body problem groups many theories concerning the relationship between the mind with its thoughts and creativity and the body producing stimuli.
One of those theories was already mentioned in Section~\ref{sec:History} with Plato and Aristotle: Dualism.
Yet, other theories are supporting Monism, a way of viewing consciousness and mind as one and not as two different components, they will be discussed as well.

The following paragraphs are accompanied by small illustrations using a \say{B} as short for \emph{Body} and a \say{M} as short for \say{Mind}.\shortfooturl[Inspired by:~]{https://en.wikipedia.org/wiki/File:Dualism-vs-Monism.png}{\formatshort{en.wikipedia.org}{Dualism-vs-Monoism.png}}{2021-02-24}


\moppreview{dualism}{Cartesian Dualism}
Many dualistic views have been discussed by a lot of philosophers in the history of humankind (e.g. Platon and Aristotle).
They all do represent roughly the same concept found in cartesian dualism, a doctrine formulated by René Descartes~\cite{Leach2017}.
Accordingly, body and mind are two different and independently existing components that causally interact with each other.
For Descartes there is an immaterial substance (\say{res cogitans}) capable of thinking and a material substance (\say{rex extensa}) incapable of thinking but responsible for physical processes.\shortfootarchiveurl{https://web.archive.org/web/20210105192001/https://iep.utm.edu/descmind/}{https://iep.utm.edu/descmind/}{2021-01-31}


\moppreview{physicalism}{Physicalism}
One monistic view on the mind-body problem is physicalism, stating that everything that exists is physical~\cite{stoljar2010physicalism}. In consequence, the mind with all of its creativity is just the result of all the physical substance constructing the brain, without anything \say{higher} at play.
In fact, physicalism goes beyond the scope of the mind-body problem,\footurl{https://plato.stanford.edu/entries/physicalism/}{2021-01-31} comparable to scientific materialism (cf. Subsection~\ref{subsec:incompatibility})~\cite{Crane1990}.

\moppreview{idealism}{Idealism}
Another monistic view is idealism which has been mentioned along with the incompatibility models favoring religion in Subsection~\ref{subsec:incompatibility}.
Idealistic perspectives consider \say{reality} as indistinguishable from human perception,
so everything that \say{exists} only does so in the mind.\shortfooturl{https://www.qcc.cuny.edu/SocialSciences/ppecorino/INTRO_TEXT/Chapter\%206\%20Mind-Body/Monism-Idealism.htm}{\formatshort{https://www.qcc.cuny.edu}{Idealism}}{2021-02-24}
Yet there are a lot of different variants.\footurl{https://plato.stanford.edu/entries/idealism/}{2021-02-24}

\moppreview{monism}{Neutral Monism}
There is not the one neutral monism, the term itself groups all theories which consider another fundamental nature to be responsible for the Mind and the Body (e.g. Ernst Mach considered this to be the combination of elements).
Baruch Spinzoza and David Hume are viewed as the originators of neutral monism~\cite{rosenkrantz2010historical}, yet their neutrality is sometimes criticized.\footurl[See 4.,~]{https://plato.stanford.edu/entries/neutral-monism/}{2021-02-24}


% Problem between body and mind https://en.wikipedia.org/wiki/File:Dualism-vs-Monism.png

% Done? \flocomment{Dualismus und Monismus sowie diverse weiter Betrachtungen. Hier vielleicht noch ne Grafik beziehungs\-weise wieder so wie im 3. Abschnitt?}

\subsection{Information technology}

{\columnsep=2ex\begin{wrapfigure}{r}{.45\linewidth}
\vspace*{-\baselineskip}
\GetAsq{information_processing}
\caption{Cognitivistic information processing.}
\vspace*{-\baselineskip}
\end{wrapfigure}
While most of the presented theories focus on philosophical aspects, they are not the only one assessing the mind-body problem.
Psychology, especially social psychology, pursues the embodiment approach~\cite{meier2012embodiment} with a cognitivistic approach: the mind needs the body to exist, yet all those stimuli generated by the body are processed as a whole and therefore not directly mapped to brain functions.\par}

With the advent of information technology and neural networks, research into a form of artificial consciousness began.
\flocomment{Warum wichtig? Wenn exakt repräsentiert \(\to\) Materialismus? Ermöglicht studien; Informationstheoretischer Ansatz in der Psychologie \(\to\) Aktueller Fortschritt. KI \& Künstliches Bewusstsein}
