\section{Consciousness}
\label{sec:Consciousness}

When talking are about science and religion and the ways they interact, there are many areas where they clash and raise conflicts to be settled: genetic engineering, nuclear science, psychological studies, and more.

A lot of those topics have been discussed extensively~\cite{barbour1993ethics,pfleiderer2010genethics,chernus1991nuclear}, but this section aims to address the concept of consciousness and artificial consciousness in particular~\cite{buttazzo2001artificial,chella2013artificial}.
Hence, the mind-body problem which deals with the role of consciousness is examined first.
Afterward, the section deals with the current state of research in the field of artificial consciousness.

\def\lmibop#1{\includegraphics[height=3\baselineskip,width=1.5cm,keepaspectratio]{figures/asq_mibop_#1-compressed.pdf}}

\newcommand*\moppreview[2]{\needspace{4\baselineskip}\paragraph[#2]{\smash{\raisebox{-2.3\baselineskip}{\lmibop{#1}}} #2}%
\parshape=4
0cm \linewidth
1.65cm \dimexpr\linewidth-1.65cm\relax
1.5cm \dimexpr\linewidth-1.5cm\relax
0cm \linewidth}

\subsection{The Mind-Body Problem}
\label{subsec:mind-body}The mind-body problem is a name, grouping many theories concerning the relationship between the mind with its thoughts and creativity and the body.
One of those theories was already mentioned in Section~\ref{sec:History} with Plato and Aristotle: Dualism.
Yet, other theories are supporting Monism, a way of viewing consciousness and mind as one and not as two different components, they will be discussed as well.

The following paragraphs are accompanied by small images using a \say{B} as short for \emph{Body} and a \say{M} as short for \say{Mind}.


\moppreview{dualism}{Cartesian Dualism}
Many dualistic views have been discussed by a lot of philosophers in the history of humankind (e.g. Platon and Aristotle).
They all do represent roughly the same concept found in cartesian dualism, a doctrine formulated by René Descartes~\cite{Leach2017}.
Accordingly, body and mind are two different and independently existing components that causally interact with each other.
For Descartes there is an immaterial substance (\say{res cogitans}) capable of thinking and a material substance (\say{rex extensa}) incapable of thinking, with the latter responsible for physical processes.\shortfootarchiveurl{https://web.archive.org/web/20210105192001/https://iep.utm.edu/descmind/}{https://iep.utm.edu/descmind/}{2021-31-01}


\moppreview{physicalism}{Physicalism}
The first monistic view on the mind-body problem is physicalism, stating that everything that exists is physical~\cite{stoljar2010physicalism}. In fact, physicalism goes beyond the scope of the mind-body problem,\footurl{https://plato.stanford.edu/entries/physicalism/}{2021-01-31} comparable to scientific materialism (cf. Subsection~\ref{subsec:incompatibility})~\cite{Crane1990}.

\moppreview{idealism}{Idealism}
Another monistic view is idealism which has been mentioned along with the incompatibility models favoring religion in Subsection~\ref{subsec:incompatibility}.
\lipsum[2]

\moppreview{monism}{Neutral Monism}
\lipsum[2]

% Problem between body and mind https://en.wikipedia.org/wiki/File:Dualism-vs-Monism.png

\flocomment{Dualismus und Monismus sowie diverse weiter Betrachtungen. Hier vielleicht noch ne Grafik beziehungs\-weise wieder so wie im 3. Abschnitt?}

\subsection{Information technology}

\flocomment{KI \& Künstliches Bewusstsein}
