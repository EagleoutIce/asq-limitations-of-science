\section{Consciousness}
\label{sec:Consciousness}

When talking about science, religion, and the ways they interact, there are many areas where they clash and raise conflicts to be settled: genetic engineering, nuclear science, psychological studies, and more.

A lot of those topics have been discussed extensively~\cite{barbour1993ethics,pfleiderer2010genethics,chernus1991nuclear}, this section aims to address the concept of consciousness and artificial consciousness in particular~\cite{buttazzo2001artificial,chella2013artificial}.
Hence, the mind-body problem which deals with the role of consciousness is presented first.
Afterwards, the section examines the current state of research in those fields. %  of artificial consciousness

\def\lmibop#1{\includegraphics[height=3\baselineskip,width=1.5cm,keepaspectratio]{figures/asq_mibop_#1-compressed.pdf}}

\newcommand*\moppreview[2]{\needspace{3\baselineskip}\paragraph[#2]{\smash{\raisebox{-2.3\baselineskip}{\lmibop{#1}}} #2}%
\parshape=4
0cm \linewidth
1.65cm \dimexpr\linewidth-1.65cm\relax
1.5cm \dimexpr\linewidth-1.5cm\relax
0cm \linewidth}

\subsection{The Mind-Body Problem}
\label{subsec:mind-body}The mind-body problem groups many theories concerning the relationship between the mind (with its thoughts and creativity) and the body (producing stimuli).
One of those theories was already mentioned in Section~\ref{sec:History} with Plato and Aristotle: Dualism.
Yet, other theories are supporting Monism, a way of viewing consciousness and mind as one and not as two different entities~-- they will be discussed as well.

The following paragraphs are accompanied by small illustrations using a \say{B} as short for \emph{Body} and a \say{M} as short for \emph{Mind}.\shortfooturl[Inspired by:~]{https://en.wikipedia.org/wiki/File:Dualism-vs-Monism.png}{\formatshort{en.wikipedia.org}{Dualism-vs-Monoism.png}}{2021-02-24}


\moppreview{dualism}{Cartesian Dualism}
Many dualistic views have been discussed by a lot of philosophers in the history of humankind (e.g. Plato and Aristotle).
They all represent roughly the same concept found in cartesian dualism, a doctrine formulated by René Descartes~\cite{Leach2017}.
Accordingly, body and mind are two different and independently existing entities that causally interact with each other.
For Descartes there is an immaterial substance (\say{res cogitans}) capable of thinking and a material substance (\say{rex extensa}) incapable of thinking but responsible for physical processes.\shortfootarchiveurl{https://web.archive.org/web/20210105192001/https://iep.utm.edu/descmind/}{https://iep.utm.edu/descmind/}{2021-01-31}


\moppreview{physicalism}{Physicalism}
One monistic view on the mind-body problem is physicalism, stating that everything that exists is physical~\cite{stoljar2010physicalism}. In consequence, the mind with all of its creativity is just the result of all the physical substance constructing the brain without anything \say{higher} at play.
In fact, physicalism goes beyond the scope of the mind-body problem,\footurl{https://plato.stanford.edu/entries/physicalism/}{2021-01-31} comparable to scientific materialism (cf.~Subsection~\ref{subsec:incompatibility})~\cite{Crane1990}.

\moppreview{idealism}{Idealism}
Another monistic view is idealism which has been mentioned along the incompatibility models favoring religion in Subsection~\ref{subsec:incompatibility}.
Idealistic perspectives consider \say{reality} as indistinguishable from human perception.
Thus, everything that \say{exists} only does so in the mind.\shortfooturl{https://www.qcc.cuny.edu/SocialSciences/ppecorino/INTRO_TEXT/Chapter\%206\%20Mind-Body/Monism-Idealism.htm}{\formatshort{https://qcc.cuny.edu}{Idealism}}{2021-02-24}
Yet, there are a lot of different variants.\footurl{https://plato.stanford.edu/entries/idealism/}{2021-02-24}

\moppreview{monism}{Neutral Monism}
There is not \say{the one} neutral monism. The term itself groups all theories which consider another fundamental nature to be responsible for the mind and the body (e.g. Ernst Mach considered this to be the combination of elements).
Baruch Spinoza and David Hume are viewed as the originators of neutral monism~\cite{rosenkrantz2010historical}, but their neutrality is sometimes criticized.\footurl[See 4.,~]{https://plato.stanford.edu/entries/neutral-monism/}{2021-02-24}

% Problem between body and mind https://en.wikipedia.org/wiki/File:Dualism-vs-Monism.png

% Done? \flocomment{Dualismus und Monismus sowie diverse weiter Betrachtungen. Hier vielleicht noch ne Grafik beziehungs\-weise wieder so wie im 3. Abschnitt?}

\subsection{The view of Information Technology}

{\columnsep=2ex\begin{wrapfigure}[11]{r}{.45\linewidth}
\vspace*{-.9\baselineskip}
\GetAsq{information_processing}
\caption{Cognitivistic information processing.}
\label{fig:info-proc-coc}
\end{wrapfigure}
While most of the presented theories focus on philosophical aspects, there are more which assess the mind-body problem.
Psychology, especially social psychology, pursues the concept of embodiment~\cite{meier2012embodiment} with a cognitivistic approach:
the mind needs the body to exist, yet all those stimuli generated by the body are processed as a whole and therefore not directly mapped to corresponding brain functions (see Figure~\ref{fig:info-proc-coc}).\par}

With the advent of information technology and neural networks, research on artificial consciousness began. If it would be possible to create consciousness solely through a programmed machine, this would not just favor physicalism but it would also undermine prominent religious views of consciousness as being something higher.

But what exactly is consciousness and what are the minimum requirements (e.g. for a machine) to be considered conscious?

\paragraph{Consciousness \& Turing} Alan Turing (Section~\ref{sec:History}) formulated his famous Turing test (originally named \say{imitation game}):\footurl{https://www1.wdr.de/wissen/technik/turing-test-100.html}{2021-03-14} a human asks questions using only a
keyboard and a monitor interfacing with two unknown partners. One of those partners is a human, while the other one is a machine. If the asking human is unable to determine who is who (machine or human), the machine should be considered to be \say{equal to the human}.

Yet, the test is heavily criticized for testing the concept of consciousness since basic heuristic principles may be enough (and may already have been enough\shortfooturl{https://www.heise.de/newsticker/meldung/Eugene-und-der-angeblich-bestandene-Turing-Test-So-einfach-nun-dann-doch-nicht-2218151.html}{\label{ftn:eugene}\formatshort{https://heise.de}{eugene}}{2021-03-14}) to fool a human into thinking that he (or she) talks to another one~\cite{john1980minds}.

\paragraph{Consciousness today} As the awareness of internal or external existence is highly subjective, there is no simple way of checking if a machine is \emph{really} self-conscious. Besides the Turing test, there is a countless number of other tests. They may all fail though, as any introspective or conscious sounding sentence can be the result of external programming or a machine which has been trained to say it.
While this might sound strange, it may be the same with humans: Solipsism\footurl{https://www.britannica.com/topic/solipsism}{2021-03-14} considers any other consciousness apart from the one of oneself to be unprovable.

Nevertheless, Baars's work~\cite{baars1993cognitive} should not go unmentioned as he lists a very convincing set of functions that have to be met to at least consider the possibility of artificial consciousness.


\paragraph{Artificial approaches}
The Turing test may have been beaten already,\footrefmark{ftn:eugene} yet there is no widely accepted self-conscious program as of today (and maybe, there never will be one~\cite[p.~231]{Meissner2020}). However, there are some promising approaches\ldots

BabyX by Soul\,Machines is their first prototype that simulates an infant. By using neural networks, BabyX can evolve during real-time face-to-face interactions in a very compelling way~\cite{Sagar2015}.

% \footurl{https://www.digitaltrends.com/cool-tech/sophia-android/}{2021-03-14}
% \shortfooturl{https://www.iflscience.com/technology/saudi-arabia-becomes-first-country-to-grant-citizenship-to-a-robot/}{\formatshort{https://iflscience.com}{sophia-citizen}}{2021-03-14}
Sophia is an adult android which has been granted citizenship in Saudi Arabia (in October 2017) and received an United Nation title just one month later.\shortfooturl{https://www.asia-pacific.undp.org/content/rbap/en/home/presscenter/pressreleases/2017/11/22/rbfsingapore.html}{\formatshort{https://asia-pacific.undp.org}{sophia-un}}{2021-03-14} While Sophia appears to be \say{alive}, such statements are heavily criticized.\shortfooturl{https://www.theverge.com/2017/11/10/16617092/sophia-the-robot-citizen-ai-hanson-robotics-ben-goertzel}{\formatshort{https://theverge.com}{sophia-critique}}{2021-03-14}

Besides those and countless other examples, some modern approaches try to use quantum computing (cf.~\cite{saxena2013}).
In \citetitle{Meissner2020}, \citeauthor{Meissner2020} gives a very well written deeper dive into the topic and presents other aspects of consciousness and difficulties in recreating them~\cite{Meissner2020}.
