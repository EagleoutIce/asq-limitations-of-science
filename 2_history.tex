\section{History}
\label{sec:History}

\begin{figure*}
    \GetAsq{timeline}
    \caption[The Timeline]{Timeline of the presented persons. Those grayed out are left as a reference for other mentions in this document (photos by wikimedia~\cite{wikimedia2021}). Rounded rectangles (\tikz[baseline]{\fill[paletteA,opacity=0.35,rounded corners=1.25pt] (0,0) rectangle (0.5,.5\baselineskip)} and \tikz[baseline]{\fill[gray,opacity=0.35,rounded corners=1.25pt] (0,0) rectangle (0.5,.5\baselineskip)}) denote the full life span of the respective person.}
    \label{fig:timeline}
\end{figure*}

This section will give a brief overview of some of the important contributions in history and how they interact with the concept of a god.
As a summary, Figure~\ref{fig:timeline} shows all the people mentioned in this section (and some more as a reference).

\paragraph{Plato}
The Athenian philosopher Plato, a disciple of Socrates, lived in ancient Greece roundabout \bce{400}.
While he is probably most famous for his school of Platonism, this document will focus on his views on consciousness and the interaction between body and mind.

Plato founded the Subject-Object-Problem, which was probably processed epistemologically by Thomas Hyde in his mind-body dualism~\cite{plato360} (see Subsection~\ref{subsec:mind-body}).
From Plato's point of view, living beings are a construct of an ephemeral body and an immortal soul, whereby the soul is the life principle and the actual self of the living being at the same time.\shortfooturl{https://www.spektrum.de/lexikon/neurowissenschaft/dualismus/3052}{\formatshort{https://spektrum.de}{dualismus}}{2021-01-17}

% [TODO: descartes? $\to$ rest with aristotle?]

\paragraph{Aristotle}
The Greek polymath and philosopher Aristotle was a student of Plato and worked directly with him and later independently.
His views on dualism are quite similar, he even indulged in the theory of multiple souls, stating different kinds: some of which die with the body, while others remain~\cite{aristotle350,hicks2015aristotle}.

In his famous work \say{Metaphysics} Aristotle writes about \say{Being}, presumably heavily influenced by Plato.
Aristotle states metaphysics as a science that takes precedence over all other sciences and that may be characterized in three ways~\cite{aristotle350}:
\begin{itemize}
    \item \emph{Ontology}, asking about what \say{Being} is (in the highest degree).
    \item \emph{Theology}, asking about existence and the unmoved mover as the primary cause for motion in the universe.
    \item \emph{Meta-science}, dealing with evidence and first principles of thought.
\end{itemize}
Yet, all the content of the metaphysics-collection, dealing with the character of definitions, identity, causality, and more, exceeds the scope of this document.
The most important consensus to be taken from it is the perception of the \mbox{mind/soul} as something divine, never being able to be explained by the sciences,
while the body (the matter) is physical and examinable.

% [TODO: Metaphysics $\to$ view of god (divine world)]

\paragraph{Copernicus and Galilei} In 1543 Nicolaus Copernicus created the heliocentric model, placing the sun at the center and all other planets of the solar system orbiting around it~\cite{copernicus1965revolutionibus}.
However, the model contradicted the old Testament's geocentric worldview,\footurl[See~]{https://biblia.com/bible/esv/joshua/10/12-13}{2021-01-23} which is why it was not widely accepted for roundabout 150 years until it was finally proven by Sir Isaac Newton.

One of the most famous representatives of the heliocentric model is Galileo Galilei, who got into a dispute with the Christian Church in the early 17\textsuperscript{th} century~\cite{folsing1983galileo}.
While the Church allowed him to speak of the heliocentric system as a hypothesis (he was even encouraged by Pope Urban \Rom{8}), Galileo's work \say{Dialogo}~\cite{galilei1632dialogo} overran the desired boundaries and \enquote*{earned} him house arrest and a teaching ban.
It was not until November 2\textsuperscript{nd} in 1992, that Galileo was rehabilitated by the Catholic Church.

% [TODO: Heliocentric world view $\to$ fight vs. church]

\paragraph{Charles Darwin} In his work \say{On the origin of species}~\cite{darwin1859origin}, Charles Robert Darwin published
 his theory of evolution in the year \citeyear{darwin1859origin} and was heavily criticized only one year later in a publication named \say{Essays and Reviews}~\cite{temple1860essays} mostly written by members of the Church of England.

 Although the theory of evolution gained acceptance in science rather quickly, it has been labeled a heresy by some (Christian) Church officials (e.g. in the aforementioned \say{Essays and Reviews}~\cite{temple1860essays}) and has left an ongoing conflict in some countries, such as the United States.
 In these countries, a not insignificant number of so-called \say{Creationists} believe in world history faithful to the bible (whereby creationism is represented in many religions~\cite{Hameed1637}) or in something called \say{Theistic Evolution}: an evolution that is compatible with religious belief and (at least from their perspective) a proof of gods design.
% TODO: criticue richard dawkins
% [TODO: Evolution Theory $\to$ gods work]

\paragraph{Alan Turing} As one of the most important code breakers during the Second World War, Alan Turing introduced one of the elementary computer models as early as 1936: the Turing machine~\cite{turing1936turing}.
While the discovery as such was already revolutionary, it laid the foundation for research in artificial intelligence, opening up a whole new perspective for human consciousness.
Furthermore, the Turing test (based on an idea by Alan Turing) is an example of numerous tests that attempt to get to the bottom of the peculiarity of the human mind.

Nowadays, a lot of people believe in a technological singularity, a hypothetical point in the (near future) where machines and artificial intelligence exceed human capacities and grow uncontrollably~\cite{Eden2013}.
Amongst those are Elon Musk and Stephen Hawking, strongly believing in the capabilities of artificial intelligence.\shortfooturl{https://www.dailydot.com/irl/superintelligence-meets-religion/}{\formatshort{https://dailydot.com}{superintelligence}}{2021-01-23}

% [TODO: Turing Machine $\to$ Neural Net $\to$ consciousness $\to$ god-given? ]

% https://en.wikipedia.org/wiki/Artificial_consciousness