\section{Conclusion}
\label{sec:Conclusion}

With Subsection~\ref{subsec:the-unknown} it may seem blunt to end with \say{we do not know, but I think\ldots} yet this is (at least in my opinion) exactly what makes this topic so interesting.
While it should be clear that one side restricting the other is not beneficial, different approaches like Levine's \say{explanatory gap}\shortfootarchiveurl{https://web.archive.org/web/20060721022750/http://www-lehre.inf.uos.de/~dbauer/stud/pom/PoM_levine.html}{\formatshort{http://www-lehre.inf.uos.de}{Levine}}{2021-03-14} show, that there are way more possible perspectives on this topic.

Furthermore, we may know, sometime in the future. The advancements in information technology make a \say{The Matrix}-scenario more and more plausible.
With artificial consciousness, we may be able to at least get further insights and thus further clues about the interplay of science and religion.

Yet, if we really do live in a snow globe will only ever be possible to discover if we do not create our own. Ethics and the resulting boundaries are important, there is no question about that (and the ethic-discussion exceeds the scope of this document).
Nevertheless, if we do not search for boundaries, we will never find them~-- and currently, we have not.

% Levine \say{Erklärungslücke}

% Umfrage Wissenschaftler \(\to\) integrationsmodell \(\to\) limits
% % https://en.wikipedia.org/wiki/Relationship_between_religion_and_science#Surveys_on_scientists_and_the_general_public_on_science_and_religion

% bezüglich der Modelle:
% Gary Ferngren, has stated: "Although popular images of controversy continue to exemplify the supposed hostility of Christianity to new scientific theories, studies have shown that Christianity has often nurtured and encouraged scientific endeavour, while at other times the two have co-existed without either tension or attempts at harmonization. If Galileo and the Scopes trial come to mind as examples of conflict, they were the exceptions rather than the rule."[64]

% => indepenent bis wir irengwannmal das große ganze verstehen, falls wir das jemals tun?
% % ich glaube nicht, dass eine höhere Macht existiert, aber ich glaube daran, dass der glaube an die Existenz einer höheren Macht psychologisch wichtig ist und viel liefern kann. (Ich)

% % kritisiere Glaube nicht.