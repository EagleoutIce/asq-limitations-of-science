\section{Discussion}
\label{sec:Discussion}This section is heavily influenced by my opinion and does not try to convey \say{the ultimate truth}. As already mentioned in Subsection~\ref{subsec:religious-views}, every individual may hold his or her \emph{own} concept of what belief or a god is. I neither intend nor want to attack anyone who has a different attitude or even strongly disagrees with my take on the topic.

In four steps, I will try to give a brief overview of my thoughts about the topic.
% Starting with some thoughts about the core problem I will continue with the (luckily sparse) authoritarian views and then focus on the relationship that I desire (and why).
% I will start with some words about the (luckily sparse) authoritarian views, then write about the relationship desired by me

\subsection{The Unknown}\label{subsec:the-unknown}
No matter how long I have thought about the relationship between science and religion and discussed it with others, it all reduces to one fact: we do not know. Science is (at least as of yet) unable to explain everything and no religion as of date was able to objectively prove the existence of anything supernatural.

Without any ultimate truth, the debate will probably be everlasting. However, scientific research has undoubtedly produced a lot of valuable insights and a lot of once-unthinkable things are mundane today.
And while I am certainly in favor of scientific research (since I study information technology), the scientifically graspable just might be a snow globe.

\subsection{The Snow Globe}\label{subsec:snow-globe}
The \say{Snow Globe} is a perspective that I have created during countless walks and in discussions with many fellow students.
It is probably more easily explained by the analogy from which the model arouse: Minecraft.\footnote{Although this works with basically any sandbox game, I have chosen to stay with the game the idea originated from.}

\paragraph{The Origin} Minecraft is a sandbox game created by Markus Persson (\say{Notch}) where the player can interact with a world consisting of blocks. While the basic rules of the game are fairly simple, they offer a lot of freedom, allowing functional computers and flying machines. Still, even today new techniques are discovered (with methods comparable to modern science) and used.\footnote{Of course, some of those discoveries merely stem from the fact that the game still receives updates, yet a lot of them hold for older versions.}

While \emph{we} know that Minecraft was programmed and that we do exist \say{outside of the game}, taking the perspective ofr an in-game character might represent the same situation as we     n, in our \say{reality}. I will coin those two states as \emph{out-game} (our reality) and \emph{in-game} (the \say{Minecraft reality}).
No matter how much research we would invest in-game, we would not know anything about the out-game world.

This perspective differs from the perspective of the \say{The Matrix}-Trilogy\footurl{https://www.imdb.com/title/tt0133093/}{2021-03-17} where the in-game state differs from the out-game state in its rules.
It further differs from Abbott's \say{Flatland}~\cite{abbott1987flatland} where the out-game state \emph{created} the in-game state (furthermore, they do not casually interact with each other).

\paragraph{The Snow Globe}
The name \say{Snow Globe} merely originated from the fact that the interaction between out- and in-game could very well be uni-directional.
The inhabitants of snow-globe-world might see the snowfall, up and down, left and right, they might be able to discover \textit{a} gravity (which might turn its direction with the snowfall),~\ldots\ yet they would be incapable of predicting the way the snow will fall as the out-game human might shake the snow globe and be interrupted by a sneeze/situational factors incomprehensible for the snow globe inhabitants.


\paragraph{Classification}
I have created this model before I knew about all of the theories presented in Section~\ref{sec:Theories}. Mapping religion as the belief of the Snow Globe inhabitants in \say{us} or at least something higher than snow-globe-world, this model could be categorized as independence (Subsection~\ref{subsec:independence}) or integration (Subsection~\ref{subsec:integration}) model as the out-game rules influence the in-game once (e.g. gravity).
I will reinforce this model in Subsection~\ref{subsec:golden-middle}.

\subsection{The Extremists}
Any intention may rot when the desire to enforce it blossoms in a one-dimensional and extremist perspective.
And while utopia (by definition) sounds great, everyone may perceive it differently\ldots
\blockquote[probably Immanuel Kant]{Die Freiheit des Einzelnen endet dort, wo die Freiheit des Anderen beginnt.\\\textcolor{gray}{One's freedom ends where the freedom of another begins.}}
Therefore, I do not like views such as the Church authority (Subsection~\ref{subsec:incompatibility}) enforcing their truth as the only one, verifiably suppressing advancements and destroying existing knowledge.

Of course, this is neither limited to the relationship between science and religion,\footnote{Yet, scientific advancements are as much a part of modern societies as religions have been just a few centuries ago.} nor is it limited to oppression from a religious side. As already mentioned in Subsection~\ref{subsec:the-unknown}: we do not know.


\subsection{The Golden middle}
\label{subsec:golden-middle}
I for myself ruled out the incompatibility model. Not just for the reasons mentioned before but for the main reason that it seems ignorant to ignore a perspective (no matter which side you are on) that you can not definitely \emph{prove} to be false.
With the other models, it is a little bit more difficult.

Furthermore, I do not believe in the dialogue model. While some dialogue is definitely of use in the most prominent topic of ethics, I do not think that this holds for any other topic. The religious \say{discoveries} and phenomena are different from scientific ones. This might be due to the current lack of understanding in the human psyche, but if so, I think this does not support the dialogue model as it would either mean \begin{orlist}
    \item a contradict free physical description of religious belief (supporting the integration model)
    \item another hint for the inexplicability of those events by scientific standards (supporting the independence model).
\end{orlist}

With this argument and the \say{Snow Globe} from Subsection~\ref{subsec:snow-globe}, I tend to support the independence or integration model: even if we can create something that appears to be artificial consciousness, consciousness as we know it might just be another artifact of \say{our} reality.

I tend to support the independence side: currently, it just seems far more plausible that we are not able to see out of our snow-globe-reality.
While ethics were in fact primarily shaped by religion (at least in the beginning), I would argue for them being merely a side effect of human evolution and therefore not solely part of the religious side.