\section{Theories}
\label{sec:Theories}
A lot of people~-- scientists and theologians alike~-- have tried to model possible relationships between science and religion.

This section takes a close look at four of the most popular models using the names coined by John Polkinghorne~\cite{barbour2000science,Polkinghorne1998,Peacocke1981}: \begin{inlist}
    \item the incompatibility model
    \item the independence model
    \item the dialogue model
    \item the integration model.
\end{inlist}
Furthermore, it will take a look at the specific perspectives of some religions.

\def\lmodel#1{\includegraphics[width=1.5cm]{figures/asq_models_#1-compressed.pdf}}
\newcommand*\modelpreview[2][0.2]{\subsection[The #2 model]{The #2 model\hfill\smash{\raisebox{-#1\baselineskip}{\lmodel{#2}}}}}
\def\shortverprefix{\thinspace\faCaretRight~}
\long\def\shortver#1{{\color{gray}
    \shortverprefix\parbox[t]{\linewidth-\widthof{\shortverprefix}}{\textit{#1}}
}}

\modelpreview{incompatibility}
\shortver{Science and Religion are fundamentally incompatible. Either there is only Religion, or only Science.}

There is not just one incompatibility model, some of them favor scientific, others favor religious views.
Yet they all view the rational approach of science as incompatible with a divine explanation and thus represent an extreme case of the science-religion-relationship.

In this document the incompatibility models are separated into two groups: \begin{inlist}
    \item models in favor of science
    \item models in favor of religion.
\end{inlist}
All of the presented models have been or are faced with harsh criticism from the other side.



\paragraph{Favouring science}
Some modern scientists (e.g. the aforementioned Stephen Hawking or the still alive Richard Dawkins) support science based-models that require no religion to suffice.
Some, like Richard Dawkins or Peter William Atkins,\footurl{https://winteryknight.com/tag/peter-atkins/}{2021-01-24} are even openly hostile and say that~\cite{dawkins2006god}: \say{\textcolor{gray}{[}religion\textcolor{gray}{]} subverts science and saps the intellect}.

All of the mentioned scientists are part of a view named \emph{scientific materialism}\shortfootarchiveurl{https://web.archive.org/web/20201128022903/https://www.sciencemeetsreligion.org/philosophy/scientific-materialism.php}{\formatshort{https://sciencemeetsreligion.org/}{materialism}}{2021-01-24} which accepts the material world as the only existing reality and denies the existence of any god or higher world.
Another view, \emph{scientific imperialism}\shortfootarchiveurl{https://web.archive.org/web/20040903060910/http://www.empireclubfoundation.com/details.asp?SpeechID=2359&FT=yes}{\formatshort{http://empireclubfoundation.com}{2359}}{2021-01-24} is a little less dismissive and accepts the existences of supernatural experiences.
Yet god is mainly used as a gap-filler and any supernatural event is to be analyzed and explained with scientific methods sooner or later~\cite{krishna1971gopi}.



rationalism vs epiricism

Peter Atkins "religion scorns, science respects"

religion collapses after certain point

-> Richard Dawkins <- oh boy oh boy, the sexi(e)st\ldots


science many ways, may look for good in nature and relflext that

Kategorienfehler => wissenschaft kann religion nicht erklären

Karl Giberson: Arroganz der Wissenschaft

\paragraph{Favouring religion}
With creationism~\cite{Hameed1637} on the one side


19 Jhd. John William Draper, Andrew Dickson White.

Wird heute heute nicht mehr so hart ausgelegt [quelle]

Gary Ferngren, has stated: "Although popular images of controversy continue to exemplify the supposed hostility of Christianity to new scientific theories, studies have shown that Christianity has often nurtured and encouraged scientific endeavour, while at other times the two have co-existed without either tension or attempts at harmonization. If Galileo and the Scopes trial come to mind as examples of conflict, they were the exceptions rather than the rule."[64]


Copernicus \& Galilelo

\modelpreview[0.3]{independence}

moderne Sicht

verschiedene Herangehensweisen für Herfahrungen => Verschiedene Ergebnisse.
Wissenscahft ist deskriptiv, Religion ist prescriptive.

=< Problem mit pythagoras

Torah "absolute Wahrheit"




Coulson und Schilling: haben viele ähnliche Vorgehensweisen
"rethoric of science

einstein kommentar: Science without religion is lame, religion without science is blind

\modelpreview[0.3]{dialogue}

Schnittmenge zwischen den beiden Gruppen

Christian scholarship vs theistic science = Wörtliche Auslegung ist ein Problem.




\modelpreview{integration}



Beeinflussen sich gegenseitig und können sich so bereichern. Viele Erkenntnisse können nebenher existieren => Gemeinsamer Repüsect
Mulsime \& Ausbau von Lehranstalten


Fokus auf Integration von \citeauthor{Barbour2002}~\cite{Barbour2002} -> Vielversprechenste



\subsection{Religious views}