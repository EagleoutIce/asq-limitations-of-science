\section{Theories}
\label{sec:Theories}
A lot of people~-- scientists and theologians alike~-- have tried to model possible relationships between science and religion.

This section takes a close look at four of the most popular models using the names coined by John Polkinghorne~\cite{barbour2000science,Polkinghorne1998,Peacocke1981}: \begin{inlist}
    \item the incompatibility model
    \item the independence model
    \item the dialogue model
    \item the integration model.
\end{inlist}
Furthermore, it will take a look at the specific perspectives of some religions.

\def\lmodel#1{\includegraphics[width=1.5cm]{figures/asq_models_#1-compressed.pdf}}
\newcommand*\modelpreview[2][0.2]{\subsection[The #2 model]{The #2 model\hfill\smash{\raisebox{-#1\baselineskip}{\lmodel{#2}}}}}
\def\shortverprefix{\thinspace\faCaretRight~}
\long\def\shortver#1{{\color{gray}
    \shortverprefix\parbox[t]{\linewidth-\widthof{\shortverprefix}}{\textit{#1}}
}}

\modelpreview{incompatibility}
\shortver{Science and Religion are fundamentally incompatible. Either there is only Religion or only Science.}

There is not just one incompatibility model, some of them favor scientific, others favor religious views.
Yet they all view the rational approach of science as incompatible with a divine explanation and thus represent an extreme case of the science-religion-relationship.

In this document the incompatibility models are separated into two groups: \begin{inlist}
    \item models in favor of science
    \item models in favor of religion.
\end{inlist}
All of the presented models have been or are heavily criticized from the other side, but their popularity has declined sharply since their peak in the 19\textsuperscript{th} century. Nowadays, a more nuanced view (as discussed with the other models) is generally favored~\cite{Ferngren2002,Jones2011}.

% 19 Jhd. John William Draper, Andrew Dickson White.

\paragraph{Favouring science}
Some modern scientists (e.g. the aforementioned Stephen Hawking or the still alive Richard Dawkins) support science based-models that require no religion to suffice.
Some, like Richard Dawkins or Peter William Atkins,\footurl{https://winteryknight.com/tag/peter-atkins/}{2021-01-24} are even openly hostile and say that~\cite{dawkins2006god}: \say{\textcolor{gray}{[}religion\textcolor{gray}{]} subverts science and saps the intellect}.

All of the mentioned scientists are part of a view named \emph{scientific materialism}\shortfootarchiveurl{https://web.archive.org/web/20201128022903/https://www.sciencemeetsreligion.org/philosophy/scientific-materialism.php}{\formatshort{https://sciencemeetsreligion.org/}{materialism}}{2021-01-24} which accepts the material world as the only existing reality and denies the existence of any god or a higher world.
Another view, \emph{scientific imperialism}\shortfootarchiveurl{https://web.archive.org/web/20040903060910/http://www.empireclubfoundation.com/details.asp?SpeechID=2359&FT=yes}{\formatshort{http://empireclubfoundation.com}{2359}}{2021-01-24} is a little less dismissive and accepts the existences of supernatural experiences.
Yet god is mainly used as a gap-filler and any supernatural event is to be analyzed and explained with scientific methods sooner or later~\cite{krishna1971gopi}.



\paragraph{Favouring religion}
The already mentioned Creationists (Section~\ref{sec:History}) regard religion as the only true perspective and belief their sacred texts (like the holy bible) word-by-word~\cite{Hameed1637}.

They are part of a view called \emph{religious fundamentalism}, which is most prominent in the united states.
A little bit less strict interpretation is named \emph{intelligent design}. It regards the world as a creation made by a divine and intelligent creator, a god.
Supported by a majority of strict believing Muslims, it rejects Darwin's theory of evolution and regards it as incompatible with the Koran~\cite{Demirci2016}.


In addition to religious fundamentalism, there is another view (which has now become rather out of date): \emph{Church authority}.
This can be found, for example, in the case of Galileo Galilei and Charles Darwin (Section \ref{sec:History}) whose findings were subordinated to the opinion of the Vatican.


% Karl Giberson: Arroganz der Wissenschaft


\modelpreview[0.3]{independence}
\shortver{Science and Religion are two different perspectives. They complement each other, but cannot be united (in a direct way).}

Similar to the category mistake,\footurl{https://plato.stanford.edu/entries/category-mistakes/}{2021-01-24} independence models (also named coexistence models) view Science and Religion as two independent languages that can not be translated into each other (easily).
While the Science-Language describes the \say{real} material world, the Religion-Language is to describe the transcendental reality.

One of the best-known representatives of this view is Albert Einstein~\cite[p.~605\thinspace ff.]{einstein1940science}:
\say{Science without religion is lame, religion without science is blind}.
Arnold Benz shares this view and proclaims that science and religion
differ in their definition of reality (objective measurements vs. experiences) and meet only at certain points, for example in the amazement and ethics.\shortfooturl{https://www.uzh.ch/about/portrait/awards/hc/2011/theol.html}{\formatshort{https://uzh.ch}{awards-2011}}{2021-01-24}

The independence model is a rather modern view and supported by the
National Academy of Sciences.\shortfooturl{https://www.nationalacademies.org/evolution/evolution-and-society}{\formatshort{https://nationalacademies.org}{evolution}}{2021-01-24}
Furthermore, it is backed by some religious people as well, e.g. Archbishop John Habgood calling science descriptive and religion perspective~\cite{habgood1964religion}.
This view is further developed by the rabbi Menachem Mendel Schneerson, who states that science, because of its arbitrary axioms, is incapable of refuting the absolute truth of the Torah.\shortfooturl{https://www.chabad.org/therebbe/letters/default_cdo/aid/66593/jewish/Torah-and-Geometry.htm}{\formatshort{https://chabad.org}{66593}}{2021-01-24} This view will be analyzed closer with the dialogue model.

\modelpreview[0.3]{dialogue}
\shortver{Science and Religion overlap in their questions. Their findings must be weighed against each other.}

As a kind of compromise, dialogue models start from overlapping fields, which use different perspectives to find common and enriched results.
Yet, at the beginning they were only sparsely represented,in favor of the other variants.
The modern view is rooted in work by Ian Barbour: \say{Myths, models, and paradigms: A comparative study in science and religion}~\cite{barbour1976myths} that re-raised the interest of several groups and focuses on ethical questions.




Christian scholarship vs theistic science = Wörtliche Auslegung ist ein Problem.




\modelpreview{integration}



Beeinflussen sich gegenseitig und können sich so bereichern. Viele Erkenntnisse können nebenher existieren => Gemeinsamer Repüsect
Mulsime \& Ausbau von Lehranstalten


Fokus auf Integration von \citeauthor{Barbour2002}~\cite{Barbour2002} -> Vielversprechenste



\subsection{Religious views}