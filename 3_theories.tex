\section{Theories}
\label{sec:Theories}
A lot of people~-- scientists and theologians alike~-- have tried to model possible relationships between science and belief.
This section takes a close look at four of the most popular models using the names coined by John Polkinghorne~\cite{barbour2000science,Polkinghorne1998,Peacocke1981}: \begin{inlist}
    \item the incompatibility model
    \item the independence model
    \item the dialogue model
    \item the integration model.
\end{inlist}
Furthermore it will take a look at the specific perspectives of some religions.

\def\modelpreview#1{\subsection[The #1 model]{The #1 model\hfill\smash{\raisebox{-0.2\baselineskip}{\includegraphics[width=1.5cm]{figures/asq_models_#1}}}}}

\modelpreview{incompatibility}

rationalism vs epiricism

Peter Atkins "religion scorns, science respects"

religion collapses after certain point

-> Richard Dawkins <- oh boy oh boy, the sexi(e)st\ldots

\paragraph{critcism}


science many ways, may look for good in nature and relflext that

Kategorienfehler => wissenschaft kann religion nicht erklären

Karl Giberson: Arroganz der Wissenschaft

\paragraph{conflict}
19 Jhd. John William Draper, Andrew Dickson White.

Wird heute heute nicht mehr so hart ausgelgt [quelle]

Gary Ferngren, has stated: "Although popular images of controversy continue to exemplify the supposed hostility of Christianity to new scientific theories, studies have shown that Christianity has often nurtured and encouraged scientific endeavour, while at other times the two have co-existed without either tension or attempts at harmonization. If Galileo and the Scopes trial come to mind as examples of conflict, they were the exceptions rather than the rule."[64]


Copernicus \& Galilelo

\modelpreview{independence}

moderne Sicht

verschiedene Herangehensweisen für Herfahrungen => Verschiedene Ergebnisse.
Wissenscahft ist deskriptiv, Religion ist prescriptive.

=< Problem mit pythagoras

Torah "absolute Wahrheit"




Coulson und Schilling: haben viele ähnliche Vorgehensweisen
"rethoric of science

einstein kommentar: Science without religion is lame, religion without science is blind

\modelpreview{dialogue}

Schnittmenge zwischen den beiden Gruppen

Christian scholarship vs theistic science = Wörtliche Auslegung ist ein Problem.




\modelpreview{integration}



Beeinflussen sich gegenseitig und können sich so bereichern. Viele Erkenntnisse können nebenher existieren => Gemeinsamer Repüsect
Mulsime \& Ausbau von Lehranstalten


Fokus auf Integration von \citeauthor{Barbour2002}~\cite{Barbour2002} -> Vielversprechenste



\subsection{Religious views}