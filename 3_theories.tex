\section{Theories}
\label{sec:Theories}
A lot of people~-- scientists and theologians alike~-- have tried to model possible relationships between science and religion.

This section takes a close look at four of the most popular models using the names coined by John Polkinghorne~\cite{Polkinghorne1998}: \begin{inlist}
    \item the incompatibility model
    \item the independence model
    \item the dialogue model
    \item the integration model
\end{inlist}
 (Others, e.g. \citeauthor{barbour2000science}~\cite{barbour2000science} and \citeauthor{Peacocke1981}~\cite{Peacocke1981} propose similar models with different names).
Furthermore, it will take a look at the specific perspectives of some religions.

\def\lmodel#1{\includegraphics[width=1.5cm]{figures/asq_models_#1-compressed.pdf}}
\newcommand*\modelpreview[2][0.2]{\needspace{4\baselineskip}\subsection[The #2 models]{The #2 models\hfill\smash{\raisebox{-#1\baselineskip}{\lmodel{#2}}}}}
\def\shortverprefix{\thinspace\faCaretRight~}
\long\def\shortver#1{{\color{gray}
    \shortverprefix\parbox[t]{\linewidth-\widthof{\shortverprefix}}{\textit{#1}}
}}

\modelpreview{incompatibility}\label{subsec:incompatibility}
\shortver{Science and religion are fundamentally incompatible. Either there is only religion or only science.}

There is not just one incompatibility model, some of them favor scientific, others favor religious views.
Yet they all view the rational approach of science as incompatible with a divine explanation and thus represent an extreme case of the science-religion-relationship.

In this document the incompatibility models are separated into two groups: \begin{inlist}
    \item models in favor of science
    \item models in favor of religion.
\end{inlist}
All of the presented models have been or are heavily criticized from the other side, but their popularity has declined sharply since their peak in the 19\textsuperscript{th} century. Nowadays, a more nuanced view (as discussed with the other models) is generally favored~\cite{Ferngren2002, Jones2011}.

% 19 Jhd. John William Draper, Andrew Dickson White.

\paragraph{Favouring science}
Some modern scientists (e.g. the aforementioned Stephen Hawking or the still alive Richard Dawkins) support science based-models that require no religion to suffice.
Some, like Richard Dawkins or Peter William Atkins,\footurl{https://winteryknight.com/tag/peter-atkins/}{2021-01-24} are even openly hostile and say that~\cite{dawkins2006god}: \say{\quoteshorten[religion] subverts science and saps the intellect}.

All of those scientists are part of a view named \emph{scientific materialism}\shortfootarchiveurl{https://web.archive.org/web/20201128022903/https://www.sciencemeetsreligion.org/philosophy/scientific-materialism.php}{\formatshort{https://sciencemeetsreligion.org/}{materialism}}{2021-01-24} which accepts the material world as the only existing reality and denies the existence of any god or a higher world.
Another view, \emph{scientific imperialism}\shortfootarchiveurl{https://web.archive.org/web/20040903060910/http://www.empireclubfoundation.com/details.asp?SpeechID=2359&FT=yes}{\formatshort{http://empireclubfoundation.com}{2359}}{2021-01-24} is a little less dismissive and accepts the existences of supernatural experiences.
Although, they are mainly used as a gap-filler and any supernatural event is to be analyzed and explained with scientific methods sooner or later~\cite{krishna1971gopi} (similar to the view of positivism). % \footurl{https://www.wissen.de/lexikon/positivismus}{2021-01-24}


\paragraph{Favouring religion}
The already mentioned Creationists (Section~\ref{sec:History}) regard religion as the only true perspective and belief their sacred texts (like the holy bible) word-by-word~\cite{Hameed1637}.

They are part of a view called \emph{religious fundamentalism}, which is most prominent in the United States.
An a little bit less strict interpretation is named \emph{intelligent design}. It regards the world as a creation made by a divine and intelligent creator, a god.
Supported by a majority of strict believing Muslims, it rejects Darwin's theory of evolution and regards it as incompatible with the Koran~\cite{Demirci2016}.


In addition to religious fundamentalism, there is another view (which has now become rather out of date): \emph{Church authority}.
This can be found, for example, in the cases of Galileo Galilei and Charles Darwin (Section \ref{sec:History}) whose findings were subordinated to the opinion of the Vatican.


In particular, in contrast to scientific materialism, there is an idealistic perspective \emph{Idealism}, in which reality is based only on human perception and exists only as some kind of spirit.\footurl{https://www.britannica.com/topic/idealism}{2021-01-31}
See Subsection~\ref{subsec:mind-body} for another view on the matter.
% TODO: \flocomment{Go beyond: mehr regeln etc (minecraft vs Welt außen).}


% Karl Giberson: Arroganz der Wissenschaft


\modelpreview[0.3]{independence}\label{subsec:independence}
\shortver{Science and religion are two different perspectives. They complement each other, but cannot be united (in a direct way).}

Similar to the category mistake,\footurl{https://plato.stanford.edu/entries/category-mistakes/}{2021-01-24} independence models (also named coexistence models) view science and religion as two independent languages that can not be translated into each other (easily).
While the \textit{Science-Language} describes the \say{real} material world, the \textit{Religion-Language} is to describe the transcendental reality.

One of the best-known representatives of this view is Albert Einstein~\cite[p.~605\,ff.]{einstein1940science}:
\say{Science without religion is lame, religion without science is blind}.
Arnold Benz shares this view and proclaims that science and religion
differ in their definition of reality (objective measurements vs. experiences) and meet only at certain points, for example in the amazement and ethics.\shortfooturl{https://www.uzh.ch/about/portrait/awards/hc/2011/theol.html}{\formatshort{https://uzh.ch}{awards-2011}}{2021-01-24}

The independence model is a rather modern view and supported by the
National Academy of Sciences.\shortfooturl{https://www.nationalacademies.org/evolution/evolution-and-society}{\formatshort{https://nationalacademies.org}{evolution}}{2021-01-24}
Furthermore, it is backed by some religious people as well, e.g. Archbishop John Habgood calling science descriptive and religion prescriptive~\cite{habgood1964religion}.
This view is further developed by the rabbi Menachem Mendel Schneerson, who states that science, because of its arbitrary axioms, is incapable of refuting the absolute truth of the Torah.\shortfooturl{https://www.chabad.org/therebbe/letters/default_cdo/aid/66593/jewish/Torah-and-Geometry.htm}{\formatshort{https://chabad.org}{66593}}{2021-01-24} This view will be analyzed further with the dialogue model coming next.

\modelpreview[0.3]{dialogue}\label{fig:dialogue}
\shortver{Science and religion overlap in their questions. Their findings must be weighed against each other.}

As a kind of compromise, dialogue models view science and religion as two overlapping fields which use different perspectives to find common and enriched results.
Yet, in the beginning, those models were only sparsely represented in favor of the other variants.
The modern view is rooted in the works of Ian Barbour: \say{Myths, models, and paradigms: A comparative study in science and religion}~\cite{barbour1976myths}, that re-raised the interest of several groups and focuses on ethical questions.

The foundations of today's ethics can be found in many ways in religions that convey ethical values and guidelines through their texts and traditions.
Therefore, there are a lot of religious perspectives that capture the value of people and their role in creation, discussions about nuclear engineering, genetic engineering, and psychological experiments exceeding the scope of this document (cf.~\cite{barbour1993ethics,reiss2001improving}).

Beyond ethical issues, the dialogue models have some problems and often require a differentiated approach:
The \emph{Church authority}-concept (cf.~Subsection~\ref{subsec:incompatibility}) has shown severe problems when either side restricts the other one (especially if they do not even allow a dialogue).
On the other hand, there are a lot of religious Nobel Prize winners~\cite{shalev2002100} and scientists, arguing for such a dialog, fully accepting scientific views like the evolution theory.\shortfooturl{https://ncse.ngo/science-and-religion-christian-scholarship-and-theistic-science}{\formatshort{https://ncse.ngo/}{religion}}{2021-01-24}

% TODO: nochmal genauer % Deutsche wikipedia hat dafür einen eigenen Abschnitt

\modelpreview{integration}\label{subsec:integration}
\shortver{Science and religion do not contradict each other. Their statements contribute to the same truth.}

From a standpoint of complex interactions, integrations models try to acknowledge mutual influences of different areas (including science and religion).
They do not just say that scientific and religious views may coexist, they emphasize them being free of any contradictions.
Perceived inconsistencies are therefore merely the consequence of a wrong or incomplete understanding.
According to Ian G. Barbour, integration models are the \say{most promising option} (of the four models presented)~\cite[p.~2]{Barbour2002}.
% NOTE: sagt auch was zu AI Seite 83 +


There are a lot of views classified as integration models (and new ones appear all the time). As a small overview, three different views are briefly highlighted below: \begin{inlist}
    \item a scientific interpretation of the Koran
    \item the process philosophy
    \item the evolution theology.
\end{inlist}

\paragraph{Koran interpretation}
Already around the 12\textsuperscript{th} century, the theologian al-Ghazālī\footnote{With full name: Abū Hāmid Muhammad ibn Muhammad al-Ghazālī.} located all knowledge (at the time) in the Koran.
He assumed that the knowledge contained in the Koran only had to be understood and thus it strengthened his belief in its divine origin.
His teachings were continued, for example, by Jalal al-Din al-Suyuti~\cite{abdurrahman2003kecsfu} in the 15\textsuperscript{th} century, who further strengthened the point of \say{all sciences} being located in the Koran.

With the 19\textsuperscript{th} century, the perspective experienced a real boom, especially with \d{T}an\d{t}āwī Jawhari's 26 volume commentary on the Koran~\cite{jawhari1932jawahir} (although it was harshly criticized for interpreting far too freely~\cite[p.~48]{Demirci2016}).
To this day, new scientific discoveries are traced back to statements in the Koran~\cite{Demirci2016}.

\paragraph{Process philosophy}
Alfred North Witehead and later his student Charles Hartshorne developed the process philosophy (later: process theology) by redefining the concept of reality.
Instead of atoms, the reality is constructed from constant change
and god is represented through creativity and order in ever-changing situations.
With this, they explain (any) God's intervention in this world by creating order in which the emerging individuals can then develop~\cite{whitehead1957process}.


\paragraph{Evolution theology}
While the creation story in Genesis\shortfootarchiveurl{https://web.archive.org/web/20210202101513/http://www.vatican.va/archive/bible/genesis/documents/bible_genesis_en.html}{\formatshort{http://vatican.va}{genesis}}{2021-03-14} seems to contradict Charles Darwin's theory of evolution if taken literally, some integration models argue for them being contradiction-free.
{\def\mto{\(\to\)}
Therefore the sequence proclaimed by Genesis: light \mto{} plants \mto{} animals \mto{} humans, is nothing more than an abstract representation (or according to the theories: verification) of Charles Darwin's theory of evolution.
}

Other variants, such as Pierre Teilhard de Chardin's theology of evolution, consider the evolution to be far from complete, striving towards a \say{point omega} that enables the unification of science and religion (the reality of the world and the reality of God~\cite{teilhard1971christianity}), which would be al-Ghazālī's idea.


\subsection{Religious views}

\label{subsec:religious-views}Up until now, the major focus lied in particular on Christianity (e.g. with the \say{Creationists}).
Nevertheless, there are~-- of course~-- a large number of other religions, some of which deal (very) differently with the topic of science or higher beings.
Therefore, this segment will briefly explain potential differences with two other religions: \begin{inlist}
    \item Hinduism, as it is said to be the oldest religion~\cite[p.~732]{Kurien2006}
    \item Buddhism, as it does not share the same conception of a god~\cite{roloff2011buddhismus}.
\end{inlist}

However, it is difficult to talk about concepts of faith without raising any conflict: every individual may hold his or her \emph{own} concept of what belief or what a god is, and they are not meant to be attacked or generalized by this brief examination.

\paragraph{Hinduism} In contrast to Christianity, Hinduism has been more open to scientific discoveries,\footurl{https://www.hinduismnet.com/hinduism_science.htm}{2021-01-31} some texts are even said to contain references supporting or underlining multiple major scientific discoveries (e.g. Einstein's Theory of Relativity).\shortfooturl{https://www.huffpost.com/entry/hinduism-science-spirituality-intersect_b_967628?guccounter=1&guce_referrer=aHR0cHM6Ly9kdWNrZHVja2dvLmNvbS8&guce_referrer_sig=AQAAABVSJMQS16DQ3vXi1Lpt__lPrEth99U2_LY3wb8ViBQpogIFTwhl0aUf__xGGdGuW9lo_s8zAPjTU_Eq6q6FLO00ybGxYH8CI5qYXQ41IE-s1QCU8JTK6nTuOcP9zqCXc-eV0J4Rj7qZlGTJaizAz8nOo8vC6bxstv9k2restybn}{\formatshort{https://huffpost.com/}{hinduism}}{2021-02-24}
This is mostly due to the fact that a lot of scientific advancements in Indian history are strongly intertwined with their religion~\cite{mitcham2005encyclopedia}.

\paragraph{Buddhism}
Especially Buddhism and Science are considered to be compatible in an extraordinary way~\cite{yong2005buddhism}.
Buddhist concepts encourage an impartial investigation of the workings of nature and most of their schools have been open to scientific discoveries~\cite{lopez2009buddhism}.
Furthermore, Buddhist practices like meditation are studied via brain-scanning and other technologies and produce invaluable insights into psychological states.\shortfooturl{https://www.bbc.com/news/world-us-canada-12661646}{\formatshort{https://bbc.com}{meditation}}{2021-01-31}